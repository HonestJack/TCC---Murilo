\documentclass[
	% -- opções da classe memoir --
	12pt,				% tamanho da fonte
	openright,			% capítulos começam em pág ímpar (insere página vazia caso preciso)
	oneside,			% para impressão em apenas anverso. Oposto a twoside
	%twoside,			% para impressão em verso e anverso. Oposto a oneside
	a4paper,			% tamanho do papel. 
	% -- opções da classe abntex2 --
	%chapter=TITLE,		% títulos de capítulos convertidos em letras maiúsculas
	%section=TITLE,		% títulos de seções convertidos em letras maiúsculas
	%subsection=TITLE,	% títulos de subseções convertidos em letras maiúsculas
	%subsubsection=TITLE,% títulos de subsubseções convertidos em letras maiúsculas
	% -- opções do pacote babel --
	english,			% idioma adicional para hifenização
	brazil				% o último idioma é o principal do documento
]{abntex2}

% Evita linhas orfãs e viúvas
\widowpenalty=10000
\clubpenalty=10000

\usepackage[subentrycounter,seeautonumberlist,nonumberlist=true]{glossaries}
%\usepackage{lmodern}			% Usa a fonte Latin Modern
\usepackage{helvet}
\renewcommand{\familydefault}{\sfdefault}
\usepackage[T1]{fontenc}		% Selecao de codigos de fonte.
\usepackage[utf8]{inputenc}		% Codificacao do documento (conversão automática dos acentos)
\usepackage{lastpage}			% Usado pela Ficha catalográfica
\usepackage{indentfirst}		% Indenta o primeiro parágrafo de cada seção.
\usepackage{color, colortbl}
% Controle das cores
\definecolor{LightCyan}{rgb}{0.88,1,1} %Cor ciano
\definecolor{Blue}{rgb}{0,0,1} %Cor ciano
\usepackage{graphicx}			% Inclusão de gráficos
\usepackage{microtype} 			% para melhorias de justificação
\usepackage{lipsum}				% para geração de dummy text
\usepackage[alf,abnt-etal-list=0,abnt-etal-cite=3,abnt-etal-text=emph]{abntex2cite}					% Citações padrão ABNT
\usepackage{tikz}
\usepackage{svg}
\usetikzlibrary{shapes,arrows,chains}
\usepackage[]{mcode}
\usepackage{multirow}
\usepackage{booktabs}
\usepackage{array}
\usepackage{longtable}
\usepackage{rotating}
\usepackage[small]{caption}
\usepackage{pbox}
\usepackage{pdfpages}
\usepackage{float}
\usepackage{amsmath, esint}
\usepackage{multicol}
\usepackage{adjustbox}
\usepackage{todonotes}
\usepackage{gensymb}
\usepackage[thinc]{esdiff}

\usepackage{soul}
\usepackage[brazil]{babel}
\usepackage{blindtext}[]

\usepackage{caption}
\usepackage{subcaption}

\newcommand{\?}{\ensuremath{\texttt{\textbf{CITATION~NEEDED}}}}
\newcommand{\oautor}{Fonte: o Autor, 2021.}



%\newcommand{\hlg}[1]{{\colorbox{green}{#1}}}
%\newcommand{\hls}[1]{{\colorbox{yellow}{#1}}}
%\newcommand{\hli}[1]{{\colorbox{lime}{#1}}}
\newcommand{\hlg}[1]{{\colorbox{white}{#1}}}
\newcommand{\hls}[1]{{\colorbox{white}{#1}}}
\newcommand{\hli}[1]{{\colorbox{white}{#1}}}


\counterwithout{equation}{chapter}

\graphicspath{
	{../Mathematica}
	{../Mathematica/Images}
	{../Classifier/Graphs}
}

%\usepackage[brazil]{babel}		% idiomas
\addto\captionsbrazil{
	%% ajusta nomes padroes do babel
	\renewcommand{\bibname}{Refer\^encias Bibliogr\'aficas}
	\renewcommand{\indexname}{\'Indice Remissivo}
	\renewcommand{\listfigurename}{Lista de Figuras}
	\renewcommand{\listtablename}{Lista de Tabelas}
	\renewcommand{\listadesiglasname}{Lista de Abreviaturas e Siglas}
	%% ajusta nomes usados com a macro \autoref
	\renewcommand{\pageautorefname}{p\'agina}
	\renewcommand{\sectionautorefname}{se{\c c}\~ao}
	\renewcommand{\subsectionautorefname}{subse{\c c}\~ao}
	\renewcommand{\paragraphautorefname}{par\'agrafo}
	\renewcommand{\subsubsectionautorefname}{subse{\c c}\~ao}
}

\newenvironment{listofabbrv}[1]{
        \chapter*{Lista de abreviaturas}
        \begin{list}{\textbf{??}}{
                \settowidth{\labelwidth}{#1}
                \setlength{\labelsep}{1em}
                \setlength{\itemindent}{0mm}
                \setlength{\leftmargin}{\labelwidth}
                \addtolength{\leftmargin}{\labelsep}
                \setlength{\rightmargin}{0mm}
                \setlength{\itemsep}{.1\baselineskip}
                \renewcommand{\makelabel}[1]{\makebox[\labelwidth][l]{##1}}
        }
}{
        \end{list}
}

\definecolor{blue}{RGB}{0,114,189}
\definecolor{orange}{RGB}{217,83,25}
\definecolor{yellow}{RGB}{237,177,32}
\definecolor{purple}{RGB}{126,47,142}
\definecolor{green}{RGB}{119,172,48}
\definecolor{lightBlue}{RGB}{77,190,238}
\definecolor{red}{RGB}{162,20,47}
\definecolor{black}{RGB}{0,0,0}

% informações do PDF
\makeatletter
\hypersetup{
     	%pagebackref=true,
		pdftitle={\@title}, 
		pdfauthor={\@author},
    	pdfsubject={\imprimirpreambulo},
	    pdfcreator={LaTeX},
		pdfkeywords={abnt}{latex}{abntex}{abntex2}{trabalho acadêmico}, 
		colorlinks=true,	% false: boxed links; true: colored links
    	linkcolor=black,	% color of internal links
    	citecolor=black,	% color of links to bibliography
    	filecolor=black,	% color of file links
		urlcolor=black,
		bookmarksdepth=4
}
\makeatother

% --- 
% Espaçamentos entre linhas e parágrafos 
% --- 
% O tamanho do parágrafo é dado por:
\setlength{\parindent}{1.3cm}
% Controle do espaçamento entre um parágrafo e outro:
\setlength{\parskip}{0.2cm}  % tente também \onelineskip

\lstset{literate=
  {á}{{\'a}}1 {é}{{\'e}}1 {í}{{\'i}}1 {ó}{{\'o}}1 {ú}{{\'u}}1
  {ã}{{\~a}}1 {õ}{{\~o}}1
  {Á}{{\'A}}1 {É}{{\'E}}1 {Í}{{\'I}}1 {Ó}{{\'O}}1 {Ú}{{\'U}}1
  {à}{{\`a}}1 {è}{{\`e}}1 {ì}{{\`i}}1 {ò}{{\`o}}1 {ù}{{\`u}}1
  {À}{{\`A}}1 {È}{{\'E}}1 {Ì}{{\`I}}1 {Ò}{{\`O}}1 {Ù}{{\`U}}1
  {ä}{{\"a}}1 {ë}{{\"e}}1 {ï}{{\"i}}1 {ö}{{\"o}}1 {ü}{{\"u}}1
  {Ä}{{\"A}}1 {Ë}{{\"E}}1 {Ï}{{\"I}}1 {Ö}{{\"O}}1 {Ü}{{\"U}}1
  {â}{{\^a}}1 {ê}{{\^e}}1 {î}{{\^i}}1 {ô}{{\^o}}1 {û}{{\^u}}1
  {Â}{{\^A}}1 {Ê}{{\^E}}1 {Î}{{\^I}}1 {Ô}{{\^O}}1 {Û}{{\^U}}1
  {œ}{{\oe}}1 {Œ}{{\OE}}1 {æ}{{\ae}}1 {Æ}{{\AE}}1 {ß}{{\ss}}1
  {ű}{{\H{u}}}1 {Ű}{{\H{U}}}1 {ő}{{\H{o}}}1 {Ő}{{\H{O}}}1
  {ç}{{\c c}}1 {Ç}{{\c C}}1 {ø}{{\o}}1 {å}{{\r a}}1 {Å}{{\r A}}1
  {€}{{\euro}}1 {£}{{\pounds}}1,
  language=Python, extendedchars=true, breaklines=true
}
