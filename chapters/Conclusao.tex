\chapter{Conclusão}

% Conclusão: resuma a solução proposta no seu projeto e as contribuições do seu trabalho.

% A solução proposta foi a realização de ensaios de estresse térmico para gerar NBTI nos transistores dos FPGAs a avaliar a degradação na frequência dos osciladores, para assim comparar a susceptibilidade dos dispositivos ao fenômeno estudado.
% O trabalho pode contribuir apresentando uma forma simples de analisar os efeitos de envelhecimento acelerado em dispositivos comerciais.



% Conclusão: descreva brevemente os resultados obtidos, fazendo uma análise crítica e comparando-os com trabalhos correlatos.

% Foi observado uma maior degradação na frequência do FPGA da placa ZedBoard (2,5\%) em relação a placa DE2 (1\%). Isso corrobora a hipótese de que as duas placas não envelhecem com a mesma intensidade, o que pode ser devido ao fato que o nó tecnológico da ZedBoard ser menor que o da DE2, o que está de acordo com a literatura, que diz que o fenômeno do NBTI é mais crítico quanto menor for o transistor.
% Comparando com outros trabalhos pode-se dizer que esses valores estão dentro do esperado. O trabalho de (Lorenz, 2013) mostra uma degradação de 5\% com 144 horas de exposição à 125°C. Já o trabalho de (Sato et al., 2014), que estudou métodos para diminuir o efeito de NBTI em osciladores em anel, resultou uma degradação de 0,25\%, com 42 horas de exposição, porém à apenas 85°C.



% Conclusão: liste, de forma justificada, possíveis trabalhos futuros decorrentes do seu projeto.

% Um possível trabalho futuro seria realizar ensaios semelhantes utilizando um FPGA de tecnologia FinFET, pois os FPGAs estudados nesse são de tecnologia CMOS planar.
% Outro trabalho poderia ser a análise de diferentes topologias de oscilador em anel, como os propostos no trabalho de (Sato et al., 2014), em FPGAs de nós tecnológicos diferentes, para verificar como essas diferentes topologias se comportam em dispositivos diferentes.
