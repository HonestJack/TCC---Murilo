\subsubsection{Mecanismos do NBTI}

De acordo com \cite{Zeng} os mecanismos físicos do NBTI podem ser explicados através de três fenômenos não relacionados, que são: a geração de armadilhas de interface, o aprisionamento de lacunas e a geração de armadilhas no óxido do bulk.

O primeiro pode ser explicado pelo modelo de reação-difusão (RD), que diz o NBTI é causado por ligações Si-H quebradas na interface entre o substrato e o oxido do gate. Essas ligações Si-H são formadas na fabricação do dispositivos para impedir que os átomos de silício fiquem com a valência incompleta após a colocação da camada de óxido de silício (SiO\small{2}) sobre o substrato. As ligações pendentes são denominadas estados de interface e podem voltar a ocorrer em condições de estresse.

A Figura \ref{fig:PmosCrossSec} mostra as ligações Si-H na interface entre o gate e o substrato de um transistor PMOS.

\begin{figure}[H]
    \centering
    \includegraphics[scale=1]{figures/Cross section of a PMOS transistor.png}
    \caption{Seção da interface gate-substrato de um transistor PMOS. Fonte: \cite{Lorenz}}
    \label{fig:PmosCrossSec}
\end{figure}

Os estados de interface resultante deterioram parâmetros do transistor. Isso pode ser modelado pelo sistema RD, composto de dois processos: uma reação local e uma difusão dos produtos da reação.

A taxa de geração dessas interfaces é dada pela Equação \ref{eq:TaxaInteface}.

\begin{equation}
    \label{eq:TaxaInteface}
    \diff{N{\textsubscript it}}{t} = K{\scriptstyle F}(N{\scriptstyle 0} - N{\scriptstyle it}) - K{\scriptstyle R}N{\scriptstyle H}(0)N{\scriptstyle it}
\end{equation}

O primeiro termo do lado direito da equação mostra a componente de geração dos estados de interface, já o segundo termo descreve a regeneração das ligações, também denominada annealing reverso, uma característica especial do NBTI.

N\textsubscript{0} representa a quantidade inicial de ligações Si-H, N\textsubscript{it} representa o número de estados de interface e K\textsubscript{R} é a taxa constante de criação de ligações quebradas. No termo de recuperação N\textsubscript{H}(0) representa o número de átomos de hidrogênio na interface do silício com o óxido, K\textsubscript{R} é a taxa constante de annealing reverso das ligações incompletas e átomos de hidrogênio em ligações Si-H.

O lado direito da equação mostra que os estados de interface voltam a diminuir quando a condição de estresse é removida.

A criação de estados de interface é limitado pela difusão dos átomos de hidrogênio, como mostrado na Equação \ref{eq:TaxaDifusao}.

\begin{equation}
    \label{eq:TaxaDifusao}
    \diff{N{\textsubscript it}}{t} = - D{\scriptstyle H}\diff{N{\textsubscript H}}{x} + N{\textsubscript H}\mu{\textsubscript H}E{\scriptstyle ox}
\end{equation}

Onde D\textsubscript{H} representa o coeficiente de difusão, $\mu$\textsubscript{H} representa a mobilidade dos átomos de hidrogênio e E\textsubscript{ox} representa o campo elétrico que atravessa o óxido.

O segundo termo pode ser negligenciado para átomos ou moléculas eletricamente neutros. K\textsubscript{F}, K\textsubscript{R} e D\textsubscript{H} dependem da temperatura. K\textsubscript{F} também depende do campo elétrico aplicado. Isso demonstra que as interfaces só são geradas quando um campo elétrico é aplicado, o que não é necessário para o annealing e para a difusão.

As Equações \ref{eq:TaxaInteface} e \ref{eq:TaxaDifusao} formam um sistema que pode ser resolvido caso seja considerado que N\textsubscript{it} é muito menor que N\textsubscript{0}. A Equação \ref{eq:ResultanteRD} mostra a solução desse sistema e a dependência da quantidade de interfaces com relação o tempo.

\begin{equation}
    \label{eq:ResultanteRD}
    N{\textsubscript {it}} = \sqrt{\frac{K{\scriptstyle F}N{\scriptstyle 0}}{2K{\scriptstyle R}}}(D{\scriptstyle H}t)^{n}
\end{equation}

Onde n representa a constante exponencial de difusão e é sempre menor que 1, de forma que a geração das interfaces irá desacelerar com o tempo.

A variação Vth será proporcional ao N\textsubscript{it}, de forma que poderá ser escrito como mostrado na Equação

Porém, esse fenômeno não explica completamente a geração de NBTI. Um segundo mecanismo relacionado é baseado no aprisionamento de lacuna em defeitos no óxido pre-existentes ou provenientes de estresse elétrico \cite{Butzen}. O campo elétrico que gerado no gate quando o PMOS está negativamente polarizado causa o tunelamento de portadoras do canal nas falhas. Esse fenômeno vem sendo cada vez mais relevante na degradação por NBTI, considerendo que falhas no óxido são mais comuns em transistores high-k.

Transistors with high-κ dielectric have higher density of pre-existing defects. From this point-of-view, hole trapping/detrapping is becoming the dominant contributor to NBTI degradation (KACZER, 2010).

Cada um desses fenômenos contribui para a variação da tensão de treashold resultando na equação \ref{eq:SomaVth}. Onde V\textsubscript{IT} á a contribuição das armadilhas de interface, V\textsubscript{HT} é a contribuição do aprisionamento em defeitos pré existentes e V\textsubscript{OT} é a contribuição do aprisionamento em defeitos gerados eletricamente.

\begin{equation}
    \label{eq:SomaVth}
    \Delta V{\textsubscript {th}} = \Delta V{\textsubscript {IT}} + \Delta V{\textsubscript {HT}} + \Delta V{\textsubscript {OT}}
\end{equation}

A Equação \ref{eq:VthTempo} mostra uma aproximação da variação da tensão de treshold ao longo do tempo considerando um nó tecnológico específico e um certo conjunto de condições ambientais \cite{Butzen}.

\begin{equation}
    \label{eq:VthTempo}
    \Delta Vth = A(TSP.t)^n
\end{equation}

Onde A é uma constante que depende da tecnologia, t é o tempo, n é a constante exponencia do NBTI e TSP é a probabilidade do transistor estar negativamente polarizado.