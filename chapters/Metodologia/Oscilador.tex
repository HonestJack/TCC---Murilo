\section{Desenvolvimento dos Osciladores em Anel}

Foram utilizadas as placas de desenvolvimento DE2 e ZedBoard. Elas foram escolhidas considerando suas disponibilidades no laboratório e por serem de fabricantes diferentes e possuírem nós tecnológicos diferentes.

A DE2 possui o FPGA XXXXX da família o Cyclone II da fabricante Altera, pertencente a Intel, que possui um nó tecnológico de 90nm. A Zedboard possui o FPGA YYYYYY, da Xilinx, agora pertencente a AMD, que possui um nó tecnológico de 28nm.

Para desenvolver e sintetizar os osciladores em anel foi utilizada a linguagem de descrição Verilog. Para o Cyclone II foi utilizada a IDE Quartus II versão 12.1, já para o ZedBoard foi utilizada a IDE Vivado versão 2023.1.

A topologia de oscilador em anel foi escolhida para os testes por sua disseminada utilização na caracterização de dispositivos MOSFET. Seu uso é amplo, pois medidas a utilizando se aproximam muito mais de aplicações reais do que medições paramétricas DC padrões.

Foram realizados testes preliminares com diferentes quantidades de inversores para encontrar uma quantidade apropriada, pois, com poucos osciladores não há tempo suficiente para os inversores chavearem e com muitos osciladores o limite de iterações que as IDEs permitem era atingido.

Considerando isso, foi decidido que em cada um dos dispositivos foi sintetizado dois osciladores, um com 1001 inversores e outro com 4999. A escolha de utilizar dois osciladores em cada FPGA foi tomada por dois motivos: para se ter certeza que as IDEs não estavam simplificando os estágios inversores do circuito sintetizado e para verificar que o envelhecimento afeta igualmente diferentes partes do FPGA.

A grande quantidade de inversores é relevante, pois assim as pequenas variações aleatórias nas características de cada transistor que compõe os dispositivos tenderão a se diluir.

Após o desenvolvimento, o circuito sintetizado foi simulado utilizando a ferramenta apropriada para cada um dos componentes. Constatada a validade da solução ela foi transferida para os FPGAs reais e será medida, através de um osciloscópio, a frequência de oscilação das saídas do circuitos.