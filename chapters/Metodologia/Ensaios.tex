\section{Ensaios de Envelhecimento}

Para induzir o fenômeno de BTI nos dispositivos foi utilizada uma câmara térmica para ensaios com componenentes eletrônicos. Inicialmente a temperatura de exposição foi de 100°C, temperatura que é inferior a faixa de ocorrência do BTI, mas que foi utilizada para verificar se os componentes não iriam ser danificados com a temperatura.

Constatando que não houve problema à 100ºC, a temperatura foi aumentada para 125°C. Como também não houve danos, a temperatura foi, então, elevada para 135°C.

Nessa temperatura foi observado que o conector de alimentação da placa DE2 apresentava sinais de derretimento, por isso, a temperatura dos ensaios foi definida em 135°C. Temperatura que está dentro da faixa de 125 e 175ºC que a bibliografia indica como sendo a faixa em que o BTI ocorre mais facilmente.

O tempo que os dispositivos forem expostas ao calor não foi contínua, devido a impossibilidade de ficar durante a noite no laboratório e, por motivos de segurança, de deixar a câmara térmica ligada sem supervisão. Portanto os dispositivos, de forma geral, foram expostos à câmara térmica durante o dia e retirados dela durante a noite, ficando ligados o tempo todo, de forma que não houvesse relaxamento.

Outras duas placas, dos mesmo modelos das ensaiadas, foram deixadas fora da câmara térmica pelo mesmo tempo, sempre ligadas e com os mesmos osciladores em anel sintetizados. Comparar as medidas entre as placas que foram aquecidas com as que nao foram é importante para verificar que se a degradação na frequência é proveniente do estresse térmico ou se é apenas resultado do funcionamento prolongado.

A câmara térmica permite realizar medidas nos dispositivos ensaiados enquanto eles estão dentro dela. Portanto foi possível medir a frequência dos osciladores em anel durante o processo de estresse térmico.

Enquanto a câmara aquece da temperatura ambiente pra temperatura alvo, as medições foram mais frequentes, de cinco em cinco minutos. Quando a temperatura alvo é atingida as medidas ficam mais esparsas.

É importante separar os efeitos instantâneos da temperatura na frequência dos osciladores do efeito a longo prazo do BTI. Por isso é necessário comparar as medidas feitas em uma mesma temperatura. O momento que a temperatura é a mais estável é quando a câmara já chegou aos 135°C.

Outra comparação relevante é a das curvas frequência por temperatura ao longo de cada ciclo.

Comparando essas medidas em relação ao tempo total estressado será possível constatar se houve ou não uma degradação na frequência de funcionamento dos dispositivos. E se houve uma diferença significativa na degradação entre as duas placas.
