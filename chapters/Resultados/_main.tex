\chapter{Resultados}
\label{sec:Resultados}

Este capítulo irá mostrar os resultados obtidos através dos dados colhidos durante os ensaios de envelhecimento detalhados na seção \ref{sec:MetEnsaios}. Serão apresentadas as curvas de como a frequência dos osciladores variou com o tempo de exposição ao calor.

\section{Medidas Iniciais}
\label{sec:ResMedidasIniciais}

As primeiras medidas foram realizadas em temperatura ambiente antes dos ensaios iniciarem, para se ter as frequências iniciais dos osciladores. A Tabela \ref{tab:FreqIniciais} mostra esses valores os dois osciladores de cada placa. Esses valores serão utilizados como valor unitário quando os valores das frequências medidas estiverem normalizado.

\begin{table}[htp]
\centering
\caption{Frequências iniciais dos osciladores.}
\begin{tabular}{|cc|cc|}
\hline
\multicolumn{2}{|c|}{\textbf{Altera DE2}} & \multicolumn{2}{c|}{\textbf{ZedBoard}} \\ \hline
\multicolumn{1}{|c|}{\textbf{1001}} & \textbf{4999} & \multicolumn{1}{c|}{\textbf{1001}} & \textbf{4999} \\ \hline
\multicolumn{1}{|c|}{1805kHz} & 364,9kHz & \multicolumn{1}{c|}{1470kHz} & 276,5kHz \\ \hline
\end{tabular}
\label{tab:FreqIniciais}
\end{table}

A frequência dos osciladores da placa ZedBoard foram menores que os da placa DE2, mesmo o nó tecnológico dela sendo menor. Isso pode ser devido ao fato das LUTs que constituem o circuito sintetizado na ZedBoard serem mais complexos que os presentes na DE2.
\section{Comparação das Curvas à temperatura ambiente}
\section{Comparação das Curvas a 135ºC}

\begin{figure}[H]
    \centering
    \includegraphics[scale=0.75]{figures/Resultados/T135DE2}
    \caption{Curva da DE2 a 135ºC. Fonte: O Autor}
    \label{fig:T135DE2}
\end{figure}

\begin{figure}[H]
    \centering
    \includegraphics[scale=0.75]{figures/Resultados/T135ZedBoard}
    \caption{Curva da ZedBoard a 135ºC. Fonte: O Autor}
    \label{fig:T135ZedBoard}
\end{figure}

\begin{figure}[H]
    \centering
    \includegraphics[scale=0.75]{figures/Resultados/T135Ambas}
    \caption{Comparação das duas placas a 135ºC. Fonte: O Autor}
    \label{fig:T135Ambas}
\end{figure}
\section{Comparação das Curvas Frequência X Temperatura}

\begin{figure}[H]
    \centering
    \includegraphics[scale=0.75]{figures/Resultados/FreqXTempDE21001}
    \caption{Curva Freqência por Temperatura do oscilador com 1001 inversores da placa DE2. Fonte: O Autor}
    \label{fig:FreqXTempDE21001}
\end{figure}

\begin{figure}[H]
    \centering
    \includegraphics[scale=0.75]{figures/Resultados/FreqXTempZedBoard1001}
    \caption{Curva Freqência por Temperatura do oscilador com 1001 inversores da placa ZedBoard. Fonte: O Autor}
    \label{fig:FreqXTempZedBoard1001}
\end{figure}
\section{Comparação das Curvas de relaxamento}

% Cite as palavras-chave que necessariamente aparecerão no capítulo sobre Experimentos e Resultados do seu Projeto de Diplomação.
% Ensaios de envelhecimento, medição de frequência, medidas de performance.

% Explique que experimentos foram (e/ou ainda serão) realizados para a obtenção de resultados que sirvam à validação da solução implementada.
% O principal experimento realizado será a exposição dos dispositivos à envelhecimento utilizando câmara térmica disponível no Laboratório de Caracterização Elétrica. Cada um dos dois FPGAs terão dois osciladores com 1001 e 4999 inversores. Eles foram expostos a temperatura de 135°C por um tempo total de 150 horas. Foi possível medir a frequência durante a exposição ao calor com um osciloscópio, obtendo-se valores de frequência para diversas temperaturas.

% Liste os materiais, métodos e ferramentas efetivamente utilizados para a execução dos ensaios e experimentos. Relacione-os com os experimentos realizados.
% Osciloscópio - Usado para medir a frequência dos osciladores.
% Câmara Térmica - Usado para estressar os FPGAs.
% FPGAs - Objeto do estresse.

% Descreva brevemente os resultados mais importantes obtidos e sua aplicação no contexto do problema.
% Os resultados mais importantes são as mudanças nos valores de frequência ao longo do tempo de envelhecimento. Foi verificado que a frequência do FPGA da placa ZedBoard degradou consideravelmente mais que a frequência do FPGA da placa DE2. Também foi verificado que o comportamento da degradação desacelera com o tempo, o que é esperado pelo que existe na literatura.

% Quais das suas hipóteses iniciais foram completa ou parcialmente comprovadas com os resultados obtidos? Quais hipóteses não foram comprovadas?
% Foi observado uma maior degradação na frequência do FPGA da placa ZedBoard. Isso corrobora a hipótese de que as duas placas não envelhecem com a mesma intensidade, o que pode ser devido ao fato que o nó tecnológico da ZedBoard ser menor que o da DE2, o que está de acordo com a literatura, que diz que o fenômeno do NBTI é mais crítico quanto menor for o transistor.

% Compare os resultados obtidos e posicione a sua solução em relação aos trabalhos correlatos apresentados no capítulo da fundamentação teórico-prática.
% Houve uma degradação de entorno de 1\% para a DE2 e 2,5\% para a ZedBoard. Comparando com outros trabalhos pode-se dizer que esses valores estão dentro do esperado.
% O trabalho de (Lorenz, 2013) mostra uma degradação de 5\% com 144 horas de exposição à 125°C.
% Já o trabalho de (Sato et al., 2014), que estudou métodos para diminuir o efeito de NBTI em osciladores em anel, resultou uma degradação de 0,25\%, com 42 horas de exposição, porém à apenas 85°C.

% Formule um resumo do conteúdo pretendido para o capítulo de experimentos e resultados, incluindo elementos utilizados para responder às perguntas anteriores.
% O capítulo de experimentos e resultados conterá as curvas de frequência por tempo de envelhecimento dos FPGAs para uma temperatura constante (Temperatura ambiente e 135°C). Será discutido as diferenças entre as duas placas e sobre o formato da curva.
% Também contará com comparação entre as curvas de frequência por temperatura de uma placa para diferentes tempos de envelhecimento e feito uma comparação entre as duas placas, sendo que a DE2 não apresentou grande alteração, já a Zedboard sim.
% Por fim, haverá uma discussão se esses resultados estão de acordo com o que foi apresentado no capítulo de Referências Bibliográficas.









