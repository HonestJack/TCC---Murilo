\section{Discussão}
\label{ResDiscussao}

De forma geral, o que foi visto nesta seção foi que o FPGA da placa ZedBoard foi mais afetado pelo estresse térmico e, consequentemente, pelo fenômeno do NBTI que o FPGA da placa DE2. Isso está dentro do estabelecido na literatura de que transistores de nós tecnológicos menores são mais propensos a sofrer do fenômeno, como visto em diversos outros trabalhos \cite{Chen}, \cite{Paul} e \cite{Zeng}.

A ZedBoard também teve uma recuperação menor da frequência degradada, mostrando que, além dos efeitos terem sidos mais severos, foram mais duradouros, o que indica que a componente permanente do NBTI \cite{Banaszeski} domina o fenômeno. De forma diferente, para a DE2 pode-se supor que existe um domínio maior da componente recuperável.

Isso pode demonstrar que, por mais que a ZedBoard seja um dispositivo mais poderoso, ela pode não ser o mais indicado para aplicações críticas que precisam operar em alta temperatura e manter uma frequência de operação constante devido a estar mais propenso aos fenômenos de envelhecimento.