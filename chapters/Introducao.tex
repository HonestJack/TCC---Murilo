%====================
% INTRODUÇÃO
	\chapter{Introdução}
%=================

Os sistemas embarcados cumprem um papel fundamental na nossa sociedade, desempenhando funções que vão de aplicações rotineiras como computadores e aparelhos celulares a aplicações críticas como tecnologia aeroespacial e médica, os sistemas estão cada vez mais complexos.

Em aplicações críticas, a confiabilidade é essencial para garantir a segurança e a saúde das pessoas. Por exemplo, em equipamentos médicos, como monitores de sinais vitais ou respiradores mecânicos, falhas podem levar a resultados graves, e até mesmo fatais \cite{Kocak}. Da mesma forma, em sistemas aeronáuticos, falhas podem resultar em acidentes com consequências desastrosas \cite{Galler}.

Com o avanço tecnológico os componentes eletrônicos estão cada vez menores, tendo sido cada vez de mais fácil acesso dispositivos cujos transistores possuem nó tecnológico na casa dos nanômetros. Esse processo é denominado miniaturização \cite{Radamson}, e ele permite a criação de circuitos mais rápidos, compactos e com menor consumo energético. Porém dispositivos menores estão mais propensos a efeitos de envelhecimento que degradam parâmetros dos transistores e diminuem sua confiabilidade em sistemas críticos \cite{Lorenz}.

% No entanto, esses sistemas embarcados muitas vezes atuam em ambientes hostis e que pode decorrer um envelhecimento acelerado e em degradação da funcionalidade ao longo do tempo. 

Entre os diversos efeitos de envelhecimento que podem afetar um transistor, um dos que tem se tornado mais crítico e alvo de extensas pesquisas é o BTI (\textit{bias-temperature instability}) \cite{Garcia}. Ele tem como consequência um aumento no valor absoluto da tensão de \textit{threshold} ($\Delta$Vth) dos transistores do tipo p e transistores do tipo n que constituem o dispositivo \cite{Paul}, o que acarreta em uma menor velocidade de transição de aberto para fechado (ou de fechado para aberto) podendo depreciar a performance do sistema como um todo.

É de grande importância entender a mudança de comportamento desses sistemas para ser possível realizar projetos com maior previsibilidade e robustez, consequentemente tornando viáveis produtos mais duráveis e seguros. Existem muitos trabalhos que realizaram estudos sobre o BTI, porém não muitos utilizando dispositivos comerciais.

Com isso, o trabalho tem como objetivo estudar os efeitos do envelhecimento em sistemas em chip de dois diferentes nós tecnológicos de CMOS planar, 90nm e 28nm, utilizando, respectivamente a placa DE2 da altera, que possui um FPGA da família Cyclone 2, e a placa ZedBoard da Xilinx, que possui um FPGA da família Artix 7.

Para isso foram definidos os seguintes objetivos específicos:
\begin{itemize}
    \item Desenvolver osciladores em anel sintetizados utilizando linguagem de descrição de hardware;
    \item Expor os componentes a envelhecimento acelerado através de uma câmera térmica;
    \item Verificar a degradação da performance;
    \item Verificar a recuperação após o estresse térmico.
\end{itemize}

% Apresentação 

% Motivação
% Miniaturização dos transistores;
% Transistores menores são mais propensos a efeitos de envelhecimento, sendo um dos mais críticos o BTI;

% Problema
% Muitos estudos sobre o efeito de BTI;
% Porém poucos que utilizam dispositivos comerciais;

% Relevância
% Um conhecimento maior sobre como o BTI afeta FPGAs comerciais permitiriam realizar uma escolha melhor no projeto de sistemas que vão atuar em ambientes críticos.

% Objetivos
% Verificar se há diferença no efeito do BTI em FPGAs de nós tecnológicos diferentes;
% Desenvolver osciladores em  anel em linguagem de descrição de hardware;
% Realizar ensaios de envelhecimento utilizando uma câmara térmica.


% Introdução: contextualize o seu trabalho e descreva sua motivação para realizá-lo.
% Com o avanço tecnológico os componentes eletrônicos estão cada vez menores, o que permite a criação de circuitos mais rápidos, compactos e com menor consumo energético. Porém transistores menores também estão mais propensos a certos fenômenos, como o NBTI, que degrada seus parâmetros. Por isso é importante compreender como esses fenômenos afetam diferentes dispositivos comerciais.

% Introdução: descreva o problema endereçado no seu projeto e a relevância de trazer uma solução a este problema.
% O principal endereçado é a falta de estudos acerca das consequências do NBTI em dispositivos comerciais. Um entendimento maior com relação a isso seria de grande utilidade na hora de se escolher um componente em um projeto crítico.

% Introdução: delimite o escopo do seu trabalho e descreva os objetivos gerais e específicos do seu projeto.
% Os ensaios serão realizados em duas placas de desenvolvimento: DE2 e ZedBoard. A DE2 possui o FPGA EP2C35F672C6N da família Cyclone II da fabricante Altera, pertencente a Intel, que possui um nó tecnológico de 90nm. A ZedBoard possui o FPGA Zynq-7000, da Xilinx, agora pertencente a AMD, que possui um nó tecnológico de 28nm.

% Introdução: descreva como a sua monografia está organizada relatando brevemente do que trata cada um dos seus capítulos.
% A monografia está organizada em cinco capítulos: Introdução, Fundamentação Teórica, Metodologia, Resultados e Conclusão.
% A Introdução apresenta a motivação e os objetivos do trabalho. A Fundamentação Teórica apresenta as bases de conhecimento para tornar o trabalho independente, ela é dividida em três seções: FPGAs, Osciladores em Anel e Efeitos de Envelhecimento. A Metodologia apresenta as ferramentas e métodos necessários para se alcançar os objetivos, ela também é dividida em três seções: Dispositivos Ensaiados, Desenvolvimento dos Osciladores em Anel e Ensaios de Envelhecimento. Os Resultados apresenta uma análise dos dados obtidos nos ensaios detalhados no capítulo anterior. Por fim, a Conclusão apresenta uma síntese dos resultados relacionando-os com a Introdução.
