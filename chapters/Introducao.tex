%====================
% INTRODUÇÃO
	\chapter{Introdução}
%=================

A cada dia que passa mais e mais sistemas embarcados estão fazendo parte de nossas vidas, desempenhando papéis que vão de aplicações rotineiras como computadores e aparelhos celulares a aplicações críticas como tecnologia aeroespacial e médica, os sistemas estão cada vez mais complexos.

Em aplicações críticas, a confiabilidade é essencial para garantir a segurança e a saúde das pessoas. Por exemplo, em equipamentos médicos, como monitores de sinais vitais ou respiradores mecânicos, falhas podem levar a resultados graves, e até mesmo fatais. Da mesma forma, em sistemas aeronáuticos, falhas podem resultar em acidentes com consequências desastrosas.

No entanto, esses sistemas embarcados muitas vezes atuam em ambientes hostis e que pode decorrer um envelhecimento acelerado e em degradação da funcionalidade ao longo do tempo.

Um dos principais fenômenos que podem ocorrer é o BTI (bias-temperature instability) que consequentemente aumenta a variação de tensão de threshold ($\Delta$Vth) dos transistores p e transistores n que constituem o dispositivo, o que acarreta em uma menor velocidade de transição de aberto para fechado (ou de fechado para aberto) podendo depreciar a performance do sistema como um todo.

Sabendo disso, é de grande importância entender a mudança de comportamento desses sistemas para ser possível realizar projetos com maior previsibilidade e robustez, consequentemente tornando viáveis produtos mais duráveis e seguros.

Com isso, o trabalho tem como objetivo estudar os efeitos do envelhecimento em sistemas em chip de, pelo menos, dois diferentes nós tecnológicos de CMOS, XXnm e YYnm.

Para isso foram definidos os seguintes objetivos específicos:
\begin{itemize}
    \item Desenvolvimento de oscilador em anel sintetizado utilizando linguagem de descrição de hardware;
    \item Exposição dos componentes a envelhecimento acelerado;
    \item Verificação na degradação da performance;
\end{itemize}
