
\documentclass[
	% -- opções da classe memoir --
	12pt,				% tamanho da fonte
	openright,			% capítulos começam em pág ímpar (insere página vazia caso preciso)
	oneside,			% para impressão em apenas anverso. Oposto a twoside
	%twoside,			% para impressão em verso e anverso. Oposto a oneside
	a4paper,			% tamanho do papel. 
	% -- opções da classe abntex2 --
	%chapter=TITLE,		% títulos de capítulos convertidos em letras maiúsculas
	%section=TITLE,		% títulos de seções convertidos em letras maiúsculas
	%subsection=TITLE,	% títulos de subseções convertidos em letras maiúsculas
	%subsubsection=TITLE,% títulos de subsubseções convertidos em letras maiúsculas
	% -- opções do pacote babel --
	english,			% idioma adicional para hifenização
	brazil				% o último idioma é o principal do documento
]{abntex2}

% Evita linhas orfãs e viúvas
\widowpenalty=10000
\clubpenalty=10000

\usepackage[subentrycounter,seeautonumberlist,nonumberlist=true]{glossaries}
%\usepackage{lmodern}			% Usa a fonte Latin Modern
\usepackage{helvet}
\renewcommand{\familydefault}{\sfdefault}
\usepackage[T1]{fontenc}		% Selecao de codigos de fonte.
\usepackage[utf8]{inputenc}		% Codificacao do documento (conversão automática dos acentos)
\usepackage{lastpage}			% Usado pela Ficha catalográfica
\usepackage{indentfirst}		% Indenta o primeiro parágrafo de cada seção.
\usepackage{color, colortbl}
% Controle das cores
\definecolor{LightCyan}{rgb}{0.88,1,1} %Cor ciano
\definecolor{Blue}{rgb}{0,0,1} %Cor ciano
\usepackage{graphicx}			% Inclusão de gráficos
\usepackage{microtype} 			% para melhorias de justificação
\usepackage{lipsum}				% para geração de dummy text
\usepackage[alf,abnt-etal-list=0,abnt-etal-cite=3,abnt-etal-text=emph]{abntex2cite}					% Citações padrão ABNT
\usepackage{tikz}
\usepackage{svg}
\usetikzlibrary{shapes,arrows,chains}
\usepackage[]{mcode}
\usepackage{multirow}
\usepackage{booktabs}
\usepackage{array}
\usepackage{longtable}
\usepackage{rotating}
\usepackage[small]{caption}
\usepackage{pbox}
\usepackage{pdfpages}
\usepackage{float}
\usepackage{amsmath, esint}
\usepackage{multicol}
\usepackage{adjustbox}
\usepackage{todonotes}
\usepackage{gensymb}
\usepackage[thinc]{esdiff}

\usepackage{soul}
\usepackage[brazil]{babel}
\usepackage{blindtext}[]

\usepackage{caption}
\usepackage{subcaption}

\newcommand{\?}{\ensuremath{\texttt{\textbf{CITATION~NEEDED}}}}
\newcommand{\oautor}{Fonte: o Autor, 2021.}



%\newcommand{\hlg}[1]{{\colorbox{green}{#1}}}
%\newcommand{\hls}[1]{{\colorbox{yellow}{#1}}}
%\newcommand{\hli}[1]{{\colorbox{lime}{#1}}}
\newcommand{\hlg}[1]{{\colorbox{white}{#1}}}
\newcommand{\hls}[1]{{\colorbox{white}{#1}}}
\newcommand{\hli}[1]{{\colorbox{white}{#1}}}


\counterwithout{equation}{chapter}

\graphicspath{
	{../Mathematica}
	{../Mathematica/Images}
	{../Classifier/Graphs}
}

%\usepackage[brazil]{babel}		% idiomas
\addto\captionsbrazil{
	%% ajusta nomes padroes do babel
	\renewcommand{\bibname}{Refer\^encias Bibliogr\'aficas}
	\renewcommand{\indexname}{\'Indice Remissivo}
	\renewcommand{\listfigurename}{Lista de Figuras}
	\renewcommand{\listtablename}{Lista de Tabelas}
	\renewcommand{\listadesiglasname}{Lista de Abreviaturas e Siglas}
	%% ajusta nomes usados com a macro \autoref
	\renewcommand{\pageautorefname}{p\'agina}
	\renewcommand{\sectionautorefname}{se{\c c}\~ao}
	\renewcommand{\subsectionautorefname}{subse{\c c}\~ao}
	\renewcommand{\paragraphautorefname}{par\'agrafo}
	\renewcommand{\subsubsectionautorefname}{subse{\c c}\~ao}
}

\newenvironment{listofabbrv}[1]{
        \chapter*{Lista de abreviaturas}
        \begin{list}{\textbf{??}}{
                \settowidth{\labelwidth}{#1}
                \setlength{\labelsep}{1em}
                \setlength{\itemindent}{0mm}
                \setlength{\leftmargin}{\labelwidth}
                \addtolength{\leftmargin}{\labelsep}
                \setlength{\rightmargin}{0mm}
                \setlength{\itemsep}{.1\baselineskip}
                \renewcommand{\makelabel}[1]{\makebox[\labelwidth][l]{##1}}
        }
}{
        \end{list}
}

\definecolor{blue}{RGB}{0,114,189}
\definecolor{orange}{RGB}{217,83,25}
\definecolor{yellow}{RGB}{237,177,32}
\definecolor{purple}{RGB}{126,47,142}
\definecolor{green}{RGB}{119,172,48}
\definecolor{lightBlue}{RGB}{77,190,238}
\definecolor{red}{RGB}{162,20,47}
\definecolor{black}{RGB}{0,0,0}

% informações do PDF
\makeatletter
\hypersetup{
     	%pagebackref=true,
		pdftitle={\@title}, 
		pdfauthor={\@author},
    	pdfsubject={\imprimirpreambulo},
	    pdfcreator={LaTeX},
		pdfkeywords={abnt}{latex}{abntex}{abntex2}{trabalho acadêmico}, 
		colorlinks=true,	% false: boxed links; true: colored links
    	linkcolor=black,	% color of internal links
    	citecolor=black,	% color of links to bibliography
    	filecolor=black,	% color of file links
		urlcolor=black,
		bookmarksdepth=4
}
\makeatother

% --- 
% Espaçamentos entre linhas e parágrafos 
% --- 
% O tamanho do parágrafo é dado por:
\setlength{\parindent}{1.3cm}
% Controle do espaçamento entre um parágrafo e outro:
\setlength{\parskip}{0.2cm}  % tente também \onelineskip

\lstset{literate=
  {á}{{\'a}}1 {é}{{\'e}}1 {í}{{\'i}}1 {ó}{{\'o}}1 {ú}{{\'u}}1
  {ã}{{\~a}}1 {õ}{{\~o}}1
  {Á}{{\'A}}1 {É}{{\'E}}1 {Í}{{\'I}}1 {Ó}{{\'O}}1 {Ú}{{\'U}}1
  {à}{{\`a}}1 {è}{{\`e}}1 {ì}{{\`i}}1 {ò}{{\`o}}1 {ù}{{\`u}}1
  {À}{{\`A}}1 {È}{{\'E}}1 {Ì}{{\`I}}1 {Ò}{{\`O}}1 {Ù}{{\`U}}1
  {ä}{{\"a}}1 {ë}{{\"e}}1 {ï}{{\"i}}1 {ö}{{\"o}}1 {ü}{{\"u}}1
  {Ä}{{\"A}}1 {Ë}{{\"E}}1 {Ï}{{\"I}}1 {Ö}{{\"O}}1 {Ü}{{\"U}}1
  {â}{{\^a}}1 {ê}{{\^e}}1 {î}{{\^i}}1 {ô}{{\^o}}1 {û}{{\^u}}1
  {Â}{{\^A}}1 {Ê}{{\^E}}1 {Î}{{\^I}}1 {Ô}{{\^O}}1 {Û}{{\^U}}1
  {œ}{{\oe}}1 {Œ}{{\OE}}1 {æ}{{\ae}}1 {Æ}{{\AE}}1 {ß}{{\ss}}1
  {ű}{{\H{u}}}1 {Ű}{{\H{U}}}1 {ő}{{\H{o}}}1 {Ő}{{\H{O}}}1
  {ç}{{\c c}}1 {Ç}{{\c C}}1 {ø}{{\o}}1 {å}{{\r a}}1 {Å}{{\r A}}1
  {€}{{\euro}}1 {£}{{\pounds}}1,
  language=Python, extendedchars=true, breaklines=true
}

% Para revisão

%% Use "final" option to remove all tracking markups
%\usepackage[final]{changes}
%\definechangesauthor[name={Valner}, color=red]{vb}
%\usepackage[final]{changes}
%\deleted[id=A]{}
%\added[id=A]{}

\titulo{
Ensaio sobre os Efeitos de Envelhecimento Acelerado Decorrentes de BTI em Circuitos Digitais
}
\autor
{
	UNIVERSIDADE FEDERAL DO RIO GRANDE DO SUL\\
	ESCOLA DE ENGENHARIA\\
	DEPARTAMENTO DE ENGENHARIA ELÉTRICA\\
	\vspace*{4\baselineskip} 
    Murilo Eduardo Reinicke
}
\local{Porto Alegre}
\data{2024}
\orientador{Prof. Dr. Tiago Roberto Balen}
\instituicao{UFRGS}
\preambulo{Projeto de Diplomação, apresentado ao Departamento de Engenharia Elétrica da Escola de Engenharia da Universidade Federal do Rio Grande do Sul, como requisito para a obtenção do grau de Engenheiro Eletricista}

% ---
% INDICE
% ---
\makeindex
% ---
% GLOSSARIO
% ---
\makeglossaries

% Exemplo de configurações do glossairo
\renewcommand*{\glsseeformat}[3][\seename]{\textit{#1}  
\glsseelist{#2}}

\begin{document}

\selectlanguage{brazil}
%\selectlanguage{english}
\frenchspacing 

\imprimircapa
\imprimirfolhaderosto*

%\begin{fichacatalografica}
%	\includepdf{FC.pdf}
%\end{fichacatalografica}

%%======================
%% FOLHA DE APROVAÇÃO
%%=====================

%\begin{folhadeaprovacao}
%	\begin{center}
%		{\ABNTEXchapterfont\large{Eduarda de Castro Guterres}}
		
%		\vspace*{\fill}
%		\begin{center}
%			\ABNTEXchapterfont\bfseries\Large\imprimirtitulo
%		\end{center}
		
%		\vspace*{\fill}
%		\hspace{.45\textwidth}
%		\begin{minipage}{.5\textwidth}
%			\imprimirpreambulo
%		\end{minipage}%
%	\end{center}
	

	%\assinatura{\textbf{Prof. Dr. Ály Ferreira Flores Filho} \\ Chefe do Departamento de Engenharia Elétrica (DELET) - UFRGS}
%	
%   
%	
 %  \begin{center}
%	BANCA EXAMINADORA
	
  %\assinatura{\textbf{Prof. Dr. Giovani Bulla} \\ UFRGS}
%	\assinatura{\textbf{Prof. Dr. Alexandre Balbinot} \\ UFRGS}
	
%	\assinatura{\textbf{Profª. Dra. Adriane Parraga} \\ UERGS}
	
%	\assinatura{\textbf{\imprimirorientador} \\ Orientador - UFRGS} 
	
 %  \end{center}
  
 %\begin{center}
%	Aprovado em 21 de Maio de 2021.
%	\end{center}
%\end{folhadeaprovacao}
%================
% AGRADECIMENTOS
%=================

%\begin{agradecimentos}
%	\blindtext
    

%\end{agradecimentos}

%==========
% Resumo / Abstract
%===========

% resumo em português
\begin{resumo}[Resumo]
% O presente trabalho tem como objetivo estudar o comportamento de FPGAs afetados por envelhecimento acelerado, mais especificamente através do fenômeno de BTI, ou Bias-Temperature Instability que ocorre em decorrência da operação do dispositivo em ambientes de alta temperatura e sobre tensão de alimentação. Consequentemente uma variação na tensão de treashold ocorre nos transistores que compõe o componente. Serão ensaiados dois FPGAs: o Cyclone II da Altera e o Kintex 7 da Xilinx, ambos de tecnologia CMOS, com nó tecnológico de, respectivamente 90nm e 28nm. Os ensaios consistirão em sintetizar circuitos osciladores em anel nos dispositivos e submete-los a alta temperatura em uma câmara térmica. Após uma quantidade a ser determinada de ciclos estresse e relaxamento a frequência de oscilação será medida e será verificado se houve uma degradação na performance dele. Também será verificado se houve uma diferença relevante entre os dois dispositivos.

    \vspace{\onelineskip}
	\noindent
        
	\textbf{Palavras-chave}: Envelhecimento Acelerado, Bias-Temperature Instability, Osciladores em Anel.

\end{resumo}

\setlength{\absparsep}{18pt} 

%=======================
% SUMÁRIOS
%========================
% inserir o sumario
\pdfbookmark[0]{\contentsname}{toc}
\tableofcontents*
\cleardoublepage

% inserir lista de ilustrações
\pdfbookmark[0]{\listfigurename}{lof}
\listoffigures*
\cleardoublepage

% inserir lista de tabelas
\pdfbookmark[0]{\listtablename}{lot}
\listoftables*
\cleardoublepage

\begin{comment} 


%insere lista de abreviaturas
\pdfbookmark[0]{\listadesiglasname}{toc}
\input{chapters/06_abbreviations.tex}
\cleardoublepage



\textual
\end{comment} 
%=====================
%CHAPTERS
%====================

% %====================
% INTRODUÇÃO
	\chapter{Introdução}
%=================

A cada dia que passa mais e mais sistemas embarcados estão fazendo parte de nossas vidas, desempenhando papéis que vão de aplicações rotineiras como computadores e aparelhos celulares a aplicações críticas como tecnologia aeroespacial e médica, os sistemas estão cada vez mais complexos.

Em aplicações críticas, a confiabilidade é essencial para garantir a segurança e a saúde das pessoas. Por exemplo, em equipamentos médicos, como monitores de sinais vitais ou respiradores mecânicos, falhas podem levar a resultados graves, e até mesmo fatais. Da mesma forma, em sistemas aeronáuticos, falhas podem resultar em acidentes com consequências desastrosas.

No entanto, esses sistemas embarcados muitas vezes atuam em ambientes hostis e que pode decorrer um envelhecimento acelerado e em degradação da funcionalidade ao longo do tempo.

Um dos principais fenômenos que podem ocorrer é o BTI (bias-temperature instability) que consequentemente aumenta a variação de tensão de threshold ($\Delta$Vth) dos transistores p e transistores n que constituem o dispositivo, o que acarreta em uma menor velocidade de transição de aberto para fechado (ou de fechado para aberto) podendo depreciar a performance do sistema como um todo.

Sabendo disso, é de grande importância entender a mudança de comportamento desses sistemas para ser possível realizar projetos com maior previsibilidade e robustez, consequentemente tornando viáveis produtos mais duráveis e seguros.

Com isso, o trabalho tem como objetivo estudar os efeitos do envelhecimento em sistemas em chip de, pelo menos, dois diferentes nós tecnológicos de CMOS planar, 90nm e 28nm.

Para isso foram definidos os seguintes objetivos específicos:
\begin{itemize}
    \item Desenvolvimento de oscilador em anel sintetizado utilizando linguagem de descrição de hardware;
    \item Exposição dos componentes a envelhecimento acelerado;
    \item Verificação na degradação da performance;
\end{itemize}

% Introdução: contextualize o seu trabalho e descreva sua motivação para realizá-lo.
% Com o avanço tecnológico os componentes eletrônicos estão cada vez menores, o que permite a criação de circuitos mais rápidos, compactos e com menor consumo energético. Porém transistores menores também estão mais propensos a certos fenômenos, como o NBTI, que degrada seus parâmetros. Por isso é importante compreender como esses fenômenos afetam diferentes dispositivos comerciais.

% Introdução: descreva o problema endereçado no seu projeto e a relevância de trazer uma solução a este problema.
% O principal endereçado é a falta de estudos acerca das consequências do NBTI em dispositivos comerciais. Um entendimento maior com relação a isso seria de grande utilidade na hora de se escolher um componente em um projeto crítico.

% Introdução: delimite o escopo do seu trabalho e descreva os objetivos gerais e específicos do seu projeto.
% Os ensaios serão realizados em duas placas de desenvolvimento: DE2 e ZedBoard. A DE2 possui o FPGA EP2C35F672C6N da família Cyclone II da fabricante Altera, pertencente a Intel, que possui um nó tecnológico de 90nm. A ZedBoard possui o FPGA Zynq-7000, da Xilinx, agora pertencente a AMD, que possui um nó tecnológico de 28nm.

% Introdução: descreva como a sua monografia está organizada relatando brevemente do que trata cada um dos seus capítulos.
% A monografia está organizada em cinco capítulos: Introdução, Fundamentação Teórica, Metodologia, Resultados e Conclusão.
% A Introdução apresenta a motivação e os objetivos do trabalho. A Fundamentação Teórica apresenta as bases de conhecimento para tornar o trabalho independente, ela é dividida em três seções: FPGAs, Osciladores em Anel e Efeitos de Envelhecimento. A Metodologia apresenta as ferramentas e métodos necessários para se alcançar os objetivos, ela também é dividida em três seções: Dispositivos Ensaiados, Desenvolvimento dos Osciladores em Anel e Ensaios de Envelhecimento. Os Resultados apresenta uma análise dos dados obtidos nos ensaios detalhados no capítulo anterior. Por fim, a Conclusão apresenta uma síntese dos resultados relacionando-os com a Introdução.

\chapter{Fundamentação Teórica e Revisão Bibliográfica}
\label{sec:Referencial}

Este capítulo trata das bases teóricas necessárias para este trabalho. Ele é dividido em três seções, cada um abordando um assunto considerado essencial para o entendimento do leitor.

A primeira sessão apresenta conceitos básicos sobre FPGAs, o que são, para que servem e como programá-los, com uma pequena explicação sobre as principais linguagens de descrição de hardware.

A segunda fala sobre osciladores em anel, explica seu funcionamento e mostra, com exemplos práticos, porque é uma topologia muito utilizada em ensaios de circuitos CMOS.

Já a terceira apresentar os efeitos de envelhecimento que afetam circuitos, focando no NBTI, que vem se tornando o principal componente na degradação da vida útil dos dispositivos.

\section{FPGAs}

FPGA, ou \textit{Field Programmable Gate Arrays}, são dispositivos eletrônicos que são baseados em matrizes de blocos lógicos configuráveis que são conectados via interconexões programáveis \cite{AmdFpga}. Eles podem ser programados e reprogramados para as funcionalidades desejadas.

Sua utilização é necessária em aplicações onde uma implementação em \textit{software} utilizando um microcontrolador não é capaz de cumprir os requisitos de frequência de operação \cite{Sulaiman}.

O que cada bloco lógico possui internamente depende da fabricante e modelo do dispositivo, mas de forma geral possuem pelo menos \textit{Look-Up Tables} (LUTs), que são implementações em hardware de tabelas verdade e elementos de memória, como Flip-Flops \cite{Sato}.

A Figura \ref{fig:FPGAStructure} ilustra a estrutura básica de um FPGA mostrando os blocos lógicos configuráveis, as conexões programáveis.

\begin{figure}[H]
    \centering
    \includegraphics[scale=0.5]{figures/ReferencialTeorico/FPGAStructure.png}
    \caption{Estrutura de um FPGA. Fonte: \cite{Sato}}
    \label{fig:FPGAStructure}
\end{figure}

Os FPGAs são, de forma geral, programados utilizando linguagens de descrição de \textit{hardware} (HDL), sendo VHDL e Verilog as mais utilizadas \cite{Ain}. Essas linguagens, diferentemente de linguagens de programação convencionais onde se escreve uma série de comandos que serão executados de maneira sequencial, descrevem um circuito elétrico que será sintetizado no dispositivo.

A linguagem VHDL é pode ser utilizada para modelar sistemas digitais em diversos níveis de abstração indo do nível de algoritmo ao nível de portas lógicas. A complexidade pode variar do mais simples ao mais complexo \cite{Wunnava}. A Figura \ref{fig:Vhdl} mostra que a linguagem VHDL também pode ser definida como uma combinação de linguagens quando consideramos o nível de abstração.

\begin{figure}[H]
    \centering
    \includegraphics[scale=0.8]{figures/ReferencialTeorico/Vhdl.png}
    \caption{Integração de linguagens que constituem o VHDL. Fonte: \cite{Wunnava}}
    \label{fig:Vhdl}
\end{figure}

% VHDL is a hardware description language employed to model a digital system or digital hardware device at many levels of abstraction, ranging from the algorithmic level to the gate level (Bhasker, 1999a). The complexity of the digital system being modeled could vary from that of a simple gate to a complete digital electronic  system,  or  anything  in  between.  The  digital  system  can  also  be  described  hierarchically.  The VHDL language can also be described as a combination of languages as shown in Figure 1.

A linguagem Verilog permite descrever sistemas digitais desde o nível de portas lógicas ao nível de algoritmo. Também descreve um design do ponto de vista comportamental, de fluxo de dados, estrutural e de atrasos \cite{Bhasker}. Além disso, define sintaxe, semântica e estrutura para realizar simulações, facilitando os testes, em nível de simulação, antes da prototipação e possui simbologia e estrutura parecida com a linguagem C, tornando-se mais familiar para quem for utilizá-la \cite{Wunnava}.

\section{Oscilador em Anel}
O oscilador em anel é uma topologia de circuito muito utilizada para a caracterização de parâmetros de circuito de diversos tipos. Um dos principais motivos para isso é a capacidade de representar uma aplicação operando em alta velocidade. Segundo \cite{Bhushan} medidas feitas sob estas condições são mais próximas das aplicações reais da tecnologia do que parametrização dc convencionais, o que é verdade principalmente para dispositivos CMOS de alta performance.

Eles também são utilizados como sensores de alta precisão que aumentam a confiabilidade de um chip, podendo ser usados para monitorar diversos parâmetros como variações de processo, temperatura e efeitos de envelhecimento \cite{Sato}. Podem ser facilmente implementados e possuem um consumo de energia pequeno.

O circuito do oscilador em anel consiste em portas lógicas inversoras ligadas em sequência com uma realimentação entre a saída da última e a entrada da primeiro. É necessário que haja um número ímpar de inversores, para assim haver uma inversão periódica da entrada e da saída. O atraso de propagação de cada inversor e a realimentação gera uma onda quadrada na saída.

O período da oscilação é duas vezes o somatório do atraso de cada inversor. A Equação \ref{eq:TotalDelay} mostra frequência de oscilação, considerando que todos os inversores têm o mesmo atraso, onde N representa o número de inversores e Ta representa o tempo de atraso de um inversor.

\begin{equation}
    F = \frac{1}{2.N.Ta}
    \label{eq:TotalDelay}
\end{equation}
         
A Figura \ref{fig:RingOsc} mostra o diagrama do oscilador em anel. O circuito em questão também possui um sinal de \textit{enable}, que, além de servir para controlar o funcionamento do oscilador, também evita que o circuito entre em um estado de metaestabilidade e não oscile.

\begin{figure}[H]
    \centering
    \includegraphics[width=\linewidth]{figures/ReferencialTeorico/RingOscModified.png}
    \caption{Oscilador em Anel. Fonte: \cite{Sparkfun}, modificado pelo autor}
    \label{fig:RingOsc}
\end{figure}

Uma quantidade na casa das centenas de inversores no circuito reduzirá as variações aleatórias intrínsecas de cada MOSFET que poderiam aparecer caso fossem medidos individualmente, permitindo uma caracterização mais confiável e robusta.

O trabalho de \cite{Bhushan} descreve estratégias de design para estruturas de osciladores em anel e também apresenta o uso dessas estruturas para mensurar consumo de energia e outros parâmetros de MOSFETs.

Um outro trabalho, \cite{Michal} realizou estudos para reduzir o consumo de energia de osciladores em anel, conseguindo isso reduzindo o número de inversores, mas acoplando capacitores a cada um deles para aumentar o atraso.
\section{Efeitos de Envelhecimento}
Os efeitos que causam envelhecimentos em circuitos podem ser divididos em dois grupos, os que causam falhas abruptas e os que causam deriva de parâmetros ao longo do tempo. Os principais exemplos do primeiro grupo são os TDDB (\textit{time-dependent dielectric breakdown}) e EM (\textit{electromigration}) \cite{Lorenz}. Já, para o segundo grupo, se tem o NBTI (\textit{negative bias temperature instability}) e o HCI (\textit{hot carrier injection}) \cite{Lorenz}.

Os efeitos do primeiro grupo devem ser tratados estocasticamente, já os do segundo grupo podem ser tratados deterministicamente \cite{Lorenz}.
       
Este trabalho tem como foco os efeitos de envelhecimento de longo prazo, portanto não será tratado sobre os efeitos que causam falhas abruptas.

\subsection{Bias-Temperature Instability}
A variação da tensão de \textit{threshold} ($\Delta$Vth) dos transistores tipo p e tipo n é a principal característica a ser levada em conta ao analisar o envelhecimento acelerado de tecnologias CMOS. Essa variação acarreta em uma menor velocidade de chaveamento dos transistores.

Uma das principais causas da variação da tensão de \textit{threshold} é o fenômeno Bias-Temperature Instability (BTI), mais especificamente o Negative BTI (NBTI) afetando os transistores do tipo p e o Positive BTI (PBTI) afetando os transistores do tipo n.

O NBTI vem sendo estudado desde a década de 60 \cite{Alam}, mas com dispositivos CMOS de nós tecnológicos cada vez menores, esse fenômeno se torna um dos principais fatores que determinam a longevidade de transistores PMOS, diferentemente de transistores NMOS, que o principal fator é o HCI \cite{Bhardwaj}.

A miniaturização dos transistores aumenta o impacto desse efeito \cite{Banaszeski}, devido a campos elétricos maiores devido a óxidos mais finos, temperaturas maiores causadas pelo aumento na densidade de transistores e o uso de dielétricos \textit{high-k}, que estão mais propensos a apresentar falhas.

Transistores PMOS possuem uma tensão de \textit{threshold} negativa. O NBTI diminui a tensão de \textit{threshold} dos transistores PMOS, portanto, aumenta o valor absoluto dela, tornando-o ainda mais negativo.

\subsubsection{Mecanismos do NBTI}

% De acordo com \cite{Zeng} os mecanismos físicos do NBTI podem ser explicados através de três fenômenos não relacionados, que são: a geração de armadilhas na interface, o aprisionamento de lacunas e a geração de armadilhas no óxido do bulk.

O NBTI não pode ser explicado por um único mecanismo físico, mas por uma superposição de diversos processos \cite{Butzen}. Dois desses fenômenos são os mais aceitos, sendo eles: a geração de armadilhas na interface, o aprisionamento de lacunas.

O primeiro pode ser explicado pelo modelo de Reação-Difusão (RD), que diz que o NBTI é causado por ligações Si-H quebradas na interface entre o substrato e o óxido do gate. Essas ligações Si-H são formadas na fabricação dos dispositivos para impedir que os átomos de silício fiquem com a valência incompleta após a colocação da camada de óxido de silício (SiO\small{2}) sobre o substrato. As ligações pendentes são denominadas estados de interface e podem voltar a ocorrer devido a campos elétricos elevados e alta temperatura \cite{Banaszeski}.

A Figura \ref{fig:PmosCrossSec} mostra as ligações Si-H na interface entre o gate e o substrato de um transistor PMOS.

\begin{figure}[H]
    \centering
    \includegraphics[scale=0.4]{figures/ReferencialTeorico/Cross section of a PMOS transistor.png}
    \caption{Seção da interface gate-substrato de um transistor PMOS. Fonte: \cite{Lorenz}}
    \label{fig:PmosCrossSec}
\end{figure}

Os estados de interface resultante deterioram parâmetros do transistor. Isso pode ser modelado pelo sistema RD, composto de dois processos: uma reação local e uma difusão dos produtos da reação.

A taxa de geração dessas interfaces é dada pela Equação \ref{eq:TaxaInteface} \cite{Lorenz}.

\begin{equation}
    \label{eq:TaxaInteface}
    \diff{N{\textsubscript it}}{t} = K{\scriptstyle F}(N{\scriptstyle 0} - N{\scriptstyle it}) - K{\scriptstyle R}N{\scriptstyle H}(0)N{\scriptstyle it}
\end{equation}

O primeiro termo do lado direito da equação mostra a componente de geração dos estados de interface, já o segundo termo descreve a regeneração das ligações, também denominada \textit{annealing} reverso, uma característica especial do NBTI.

$N\scriptstyle{0}$ representa a quantidade inicial de ligações Si-H, $N\scriptstyle{it}$ representa o número de estados de interface e $K\scriptstyle{R}$ é a taxa constante de criação de ligações quebradas. No termo de recuperação $N\scriptstyle{H}(0)$ representa o número de átomos de hidrogênio na interface do silício com o óxido, $K\scriptstyle{R}$ é a taxa constante de \textit{annealing} reverso das ligações incompletas e átomos de hidrogênio em ligações Si-H.

O lado direito da equação mostra que os estados de interface voltam a diminuir quando a condição de estresse, causada pela alta temperatura e o campo elétrico, é removida.

A criação de estados de interface é limitado pela difusão dos átomos de hidrogênio, como mostrado na Equação \ref{eq:TaxaDifusao}.

\begin{equation}
    \label{eq:TaxaDifusao}
    \diff{N{\scriptstyle it}}{t} = - D{\scriptstyle H}\diff{N{\scriptstyle H}}{x} + N{\scriptstyle H}\mu{\scriptstyle H}E{\scriptstyle ox}
\end{equation}

Onde $D\scriptstyle{H}$ representa o coeficiente de difusão, $\mu\scriptstyle{H}$ representa a mobilidade dos átomos de hidrogênio e $E\scriptstyle{ox}$ representa o campo elétrico que atravessa o óxido.

O segundo termo pode ser negligenciado para átomos ou moléculas eletricamente neutros \cite{Lorenz}. $K\scriptstyle{F}$, $K\scriptstyle{R}$ e $D\scriptstyle{H}$ dependem da temperatura. $K\scriptstyle{F}$ também depende do campo elétrico aplicado. Isso demonstra que as interfaces só são geradas quando um campo elétrico é aplicado, o que não é necessário para o \textit{annealing} e para a difusão.

As Equações \ref{eq:TaxaInteface} e \ref{eq:TaxaDifusao} formam um sistema que pode ser resolvido caso seja considerado que $N\scriptstyle{it}$ é muito menor que $N\scriptstyle{0}$. A Equação \ref{eq:ResultanteRD} mostra a solução desse sistema e a dependência da quantidade de interfaces com relação o tempo.

\begin{equation}
    \label{eq:ResultanteRD}
    N{\scriptstyle it} = \sqrt{\frac{K{\scriptstyle F}N{\scriptstyle 0}}{2K{\scriptstyle R}}}(D{\scriptstyle H}t)^{n}
\end{equation}

Onde n representa a constante exponencial de difusão e é sempre menor que 1, de forma que a geração das interfaces irá desacelerar com o tempo.

A variação Vth será proporcional ao $N\textsubscript{it}$, de forma que poderá ser escrito como mostrado na Equação \ref{eq:VthProp}, onde $\Phi\scriptstyle{S}$ é o potencial de superfície e $C{\scriptstyle ox}$ é a capacitância do óxido.

\begin{equation}
    \label{eq:VthProp}
    V{\scriptstyle th} \propto - \frac{qN{\scriptstyle it}(\Phi{\scriptstyle S})}{C{\scriptstyle ox}}
\end{equation}

Porém, esse fenômeno não explica completamente o fenômeno, não descrevendo corretamente a recuperação rápida que ocorre quando as condições de estresse não estão mais presentes \cite{Gilson}.

Um segundo mecanismo relacionado, denominado \textit{Trapping/Detrapping}, é baseado no aprisionamento de lacunas em defeitos no óxido preexistentes ou provenientes de estresse elétrico \cite{Butzen}. O campo elétrico que é gerado no gate quando o PMOS está negativamente polarizado causa o tunelamento de portadores do canal, levando-as diretamente até nas falhas no óxido. Esse fenômeno vem sendo cada vez mais relevante na degradação por NBTI, considerando que falhas no óxido são mais comuns em transistores \textit{high-k}.

Cada armadilha apresenta como característica a probabilidade de capturar e liberar um portador e o valor do impacto na tensão de \textit{threshold} será gerado em caso de captura. As probabilidades tem relação com os tempos médios entre as capturas e emissões, já o impacto na tensão de \textit{threshold} tem relação com a localização em que a armadilha está localizada, podendo ser muito relevante caso obstrua o caminho de percolação do canal. 

% O modelo Trapping/Detrapping foi inicialmente desenvolvido para explicar a rápida recuperação do efeito de BTI. Assim surgiu um modelo misto, onde dois modelos coexistem: criação de defeitos na interface, dado pelo modelo Reaction-Difusion, e a captura e liberação de cargas por armadilhas pré-existentes no interior do óxido do transistor (HUARD, 2007). Neste modelo, a criação de defeitos na interface é o responsável pela parte permanente, não recuperável, da degradação enquanto as armadilhas, pré-existentes no interior do óxido, seriam responsáveis pela parte recuperável de BTI.

% Modelos mais recentes, (GRASSER, 2009) (KACZER, 2009), propõem que BTI pode ser explicado unicamente pelo efeito de captura e liberação de portadores no interior do dielétrico.

% No modelo Trapping/Detrapping, é explicada a variação de Vth pela pré-existência de armadilhas no interior do óxido, na qual cada armadilha tem como propriedades as probabilidades de captura e emissão, assim como, o desvio que causará no Vth. As probabilidades de captura e emissão são dadas pelo de tempo de captura e pelo tempo de emissão, que são os tempos médios transcorridos para que a armadilha capture um portador e emita esse portador, respectivamente. Esses tempos são log-uniformimente distribuídos, ou seja, armadilhas com diversas ordens de grandezas diferentes podem ser encontradas com a mesma probabilidade no óxido do transistor (GRASSER, 2010).

% De acordo com o modelo, quando uma armadilha é ocupada por um portador de carga (elétron ou lacuna) ela ficará eletricamente carregada. Assim, a tensão de limiar do transistor é alterada, de acordo com a localização da armadilha nas três dimensões do óxido (espessura, largura e comprimento). A localização na espessura do óxido determina o efeito eletrostático que a armadilha terá na porta e no canal de inversão do transistor, já a posição da armadilha na largura e no comprimento do óxido poderá determinar quão impactante será seu efeito na mobilidade dos portadores no transistor, já que o seu efeito eletrostático pode obstruir um caminho de percolação, ou percolation-path, no canal (KACZER, 2010) (Ver Seção 2.5), diminuindo drasticamente a sua condutividade. A localização da armadilha no comprimento do canal também determina o impacto da armadilha na tensão superficial ao longo da fonte e do dreno. Combinando todos os efeitos, dados pela localização da armadilha no canal, tem-se o desvio no Vth dado por uma determinada armadilha.

% Durante o período de estresse, armadilhas com diferentes tempos de captura e diferentes desvios de Vth são povoadas, alterando assim, a tensão limiar ao longo do tempo, conforme ilustrado na Figura 2.7, a seguir.


% Transistors with high-κ dielectric have higher density of pre-existing defects. From this point-of-view, hole trapping/detrapping is becoming the dominant contributor to NBTI degradation (KACZER, 2010).

Caminhos de percolação são os caminhos de condução que se formam em transistores nanométricos nos lugares onde há a falta de átomos dopantes, pois estes formam barreiras onde a energia de condução é maior, bloqueando os portadores \cite{Ashraf}.

% É sabido que a região onde se encontra o átomo dopante é uma região de maior resistência para a passagem do portador, funcionando como barreiras onde a energia de condução é mais elevada. Assim, em transistores nanométricos, onde não há presença de dopantes formam-se caminhos definidos pelo qual ocorre a condução da corrente, chamados de caminho de percolação, ou percolation-path (ASHRAF, 2011).

Esses dois mecanismos podem ser considerados simultaneamente, sendo o primeiro responsável por uma componente permanente da degradação e o segundo responsável por uma componente recuperável da degradação \cite{Banaszeski}.

Cada um desses fenômenos contribui para a variação da tensão de threshold resultando na Equação \ref{eq:SomaVth}. Onde $\Delta V\scriptstyle{IT}$ é a contribuição das armadilhas na interface (primeiro mecanismo), $\Delta V\scriptstyle{HT}$ é a contribuição do aprisionamento em defeitos pré existentes e $\Delta V\scriptstyle{OT}$ é a contribuição do aprisionamento em defeitos gerados eletricamente (segundo mecanismo).

\begin{equation}
    \label{eq:SomaVth}
    \Delta V{\scriptstyle {th}} = \Delta V{\scriptstyle {IT}} + \Delta V{\scriptstyle {HT}} + \Delta V{\scriptstyle {OT}}
\end{equation}

A Equação \ref{eq:VthTempo} mostra uma aproximação da variação da tensão de threshold ao longo do tempo considerando um nó tecnológico específico e um certo conjunto de condições ambientais \cite{Butzen}.

\begin{equation}
    \label{eq:VthTempo}
    \Delta Vth = A(TSP.t)^n
\end{equation}

Onde \textit{A} é uma constante que depende da tecnologia, \textit{t} é o tempo, \textit{n} é a constante exponencial do NBTI e \textit{TSP} é a probabilidade do transistor estar negativamente polarizado.

O \textit{TSP} de um transistor pode depender não só de seu duty cicle, mas também de sua posição na topologia do circuito. Por exemplo, no caso de dois transistores que compõe a malha de pull-up de uma porta NOR, o TSP do transistor superior depende apenas de seu sinal de entrada, já o TSP do inferior depende da combinação de seu sinal de entrada e do sinal de entrada do transistor superior, de forma que o transistor superior terá uma tendência maior a sofrer envelhecimento.

Quando não há campo elétrico atravessando o óxido do transistor, o que no caso dos PMOS ocorre quando a tensão de gate é positiva, há a recuperação das armadilhas de interface \cite{Chen}. Ou seja, em um circuito em que um transistor alterna entre fechado e aberto, haverá uma certa recuperação da degradação quando o transistor estiver fechado. A Figura \ref{fig:recover} exemplifica esse comportamento de estresse e relaxamento em um transistor.

\begin{figure}[H]
    \centering
    \includegraphics[scale=0.8]{figures/ReferencialTeorico/Recover.png}
    \caption{Estresse e relaxamento de um transistor. Fonte: \cite{Chen}}
    \label{fig:recover}
\end{figure}

\subsubsection{Gerando NBTI}
\label{sec:GerandoNbti}
Muitos estudos já foram realizados para determinar em que condições o NBTI ocorre com mais facilidade em circuitos CMOS.

O fenômeno NBTI normalmente ocorre em transistores do tipo p operando com tensão de gate negativa em temperaturas variando de 100ºC a 250ºC \cite{Davidovic}. Os campos elétricos devem ser na faixa dos 6MV/cm, valores encontrados durante o burn-in do componente, porém com transistores cada vez menores, esses campos podem ocorrer durante a operação normal de dispositivos de alta performance \cite{Schroder}. A Figura \ref{fig:CampoEletricoAno} mostra o aumento do campo elétrico que atravessa o óxido em transistores CMOS ao longo dos anos.

\begin{figure}[H]
    \centering
    \includegraphics[scale=0.5]{figures/ReferencialTeorico/CampoEletricoAno.png}
    \caption{Campo elétrico no óxido em dispositivos CMOS ao longo dos anos. Fonte: \cite{Schroder}}
    \label{fig:CampoEletricoAno}
\end{figure}

% Typical stress temperatures lie in the 100– 250 °C range with oxide electric fields typically below 6 MV/cm, i.e., fields below those that lead to hot carrier degradation. Such fields and temperatures are typically encountered during burn in, but are also approached in highperformance ICs during routine operation.

Para aplicar o efeito de NBTI no dispositivo ensaiado no trabalho \cite{Davidovic}, os pesquisadores o estressaram por 2000 horas, aplicando tensões negativas de 30 a 45V no gate (com fonte e dreno aterrados) em uma temperatura de variando de 125 a 175ºC.

O trabalho de \cite{Bhardwaj} desenvolveu um modelo preditivo para NBTI em dispositivos CMOS de nó tecnológico de 45nm, que alcançou estimativas precisas da degradação em longo prazo da tensão de threshold de transistores PMOS devido ao fenômeno.

Um outro trabalho \cite{Grossi}, realizou simulações para analisar os efeitos do BTI em amplificadores operacionais, e viu que o ganho DC, a frequência de corte e o slew rate são significativamente degradados em AMPOPs operando em malha aberta. Já para Ampops operando com realimentação negativa apenas a frequência de corte mostrou uma degradação significativa.

Alguns trabalhos, inclusive, realizaram estudos dos efeitos de NBTI em osciladores em anel. Um deles \cite{Lorenz} mostra uma degradação de 5\% com 144 horas de exposição à 125°C. Já outro \cite{Sato}, que estudou métodos para diminuir o efeito de NBTI em osciladores em anel, resultou uma degradação de 0,25\%, com 42 horas de exposição, porém à apenas 85°C. Um terceiro trabalho \cite{Linder} propõe topologias de osciladores em anel que permitem estudar em separado os efeitos do PBTI, nele foi medida um degradação de 1,8\% considerando apenas o PBTI, 2,2\% considerando apenas o NBTI e 3,9\%  considerando apenas os dois efeitos combinados tendo sido realizado um estresse de 2 horas e 47 minutos (10000 segundos) segundos à 125°C.


% \subsection{Hot Carrier Injection}

O fenômeno conhecido como Hot Carrier Injection (HCI) vem sendo considerado uma preocupação no funcionamento de transistores MOS desde a década de 70 e vem sendo um dos efeitos de envelecimento mais estudados desde então \cite{Butzen}.

A explicação clássica para o mecanismo físico desse efeitos é a geração de portadoras de alta energia devido a campos elétricos laterais elevados em transistores MOS quando estão no estado de saturação. Se essas portadoras ganharem energia suficiente podem ultrapassar a barreira de potencial e ser injetada no óxido do gate, causando danos na interface entre o óxido e o silício, aumentando a densidade de interfaces de estado e, consequentemente degradando parâmetros do componente, entre eles a tensão de treshold \cite{Cacho}.

O HCI causa um impacto muito maior em transistores NMOS em comparação aos transistores PMOS devido ao fato da mobilidade de portadores do tipo n ser, de forma geral, maior do que a mobilidade de portadores do tipo p \cite{Jiang}.

A Figura \ref{fig:hci} mostra Hot Carriers gerados por um alto campo elétrico lateral.

\begin{figure}[H]
    \centering
    \includegraphics[scale=0.5]{figures/ReferencialTeorico/HCI.png}
    \caption{Geração de Hot Carriers (acima) devido a um campo elétrico leteral (abaixo). Fonte: \cite{Jiang}}
    \label{fig:hci}
\end{figure}



\chapter{Metodologia}
\label{sec:Metodologia}
Neste capítulo são descritos os procedimentos utilizados para verificar os efeitos de envelhecimento acelerado nos FPGAs de interesse. Ele será composto por três seções.

A primeira apresenta os FPGAs escolhidos para serem estudados, além dos motivos para essa escolha e de características desses dispositivos.

A segunda descreve como os osciladores em anel foram projetados, programados e sintetizados nos FPGAs, além de detalhar sobre as decisões tomadas com relação a topologia e número de inversores dos osciladores.

Por fim, a terceira detalha os ensaios realizados utilizando a câmara térmica para estressar os dispositivos e envelhecê-los, bem como as análises e comparações realizadas para alcançar os resultados.

\section{Dispositivos Ensaiados}
\label{sec:MetDispositivos}

Foram utilizadas as placas de desenvolvimento DE2 e ZedBoard. Elas foram escolhidas considerando suas disponibilidades no laboratório e por serem de fabricantes diferentes e possuírem nós tecnológicos diferentes.

A DE2 possui o FPGA EP2C35F672C6N da família o Cyclone II da fabricante Altera, pertencente a Intel, que possui um nó tecnológico de 90nm. A ZedBoard possui o SoC Zynq-7000, da Xilinx, agora pertencente a AMD, que contém um processardor ARM Cortex-A9 de dois núcleos e um FPGA Artix 7 de nó tecnológico de 28nm \cite{AmdFpga2}.

Ambas as placas possuem, além dos FPGAs, componentes e periféricos necessários para testes e prototipação de sistemas, como: botões, chaves, LEDs, display, memória flash e diversas entradas e saídas.

As Figuras \ref{fig:DE2Board} e \ref{fig:ZedBoard} mostram, respectivamente, a placa de desenvolvimento DE2 e ZedBoard utilizadas.

\begin{figure}[H]
    \centering
    \includegraphics[scale=0.2]{figures/Metodologia/DE2.jpeg}
    \caption{Placa DE2. Fonte: O Autor}
    \label{fig:DE2Board}
\end{figure}

\begin{figure}[H]
    \centering
    \includegraphics[scale=0.2]{figures/Metodologia/ZedBoard1.jpeg}
    \caption{Placa ZedBoard. Fonte: O Autor}
    \label{fig:ZedBoard}
\end{figure}

% EP2C35F672C6N K CBC9Y0843A
% https://community.element14.com/products/devtools/technicallibrary/w/documents/10126/altera-cyclone-fpga-series-overview

% The Zynq™ 7000 SoC family integrates the software programmability of an ARM®-based processor with the hardware programmability of an FPGA, enabling key analytics and hardware acceleration while integrating CPU, DSP, ASSP, and mixed signal functionality on a single device. Consisting of single-core Zynq 7000S and dual-core Zynq 7000 devices, the Zynq 7000 family offers an exceptional price to performance-per-watt, fully scalable SoC platform for your unique application requirements.
% Zynq 7000 devices are equipped with dual-core ARM Cortex-A9 processors integrated with 28nm Artix 7 or Kintex™ 7 based programmable logic for excellent performance-per-watt and maximum design flexibility. With up to 6.6M logic cells and offered with transceivers ranging from 6.25Gb/s to 12.5Gb/s, Zynq 7000 devices enable highly differentiated designs for a wide range of embedded applications including multi-camera drivers assistance systems and 4K2K Ultra-HDTV.
% https://www.xilinx.com/products/silicon-devices/soc/zynq-7000.html
\section{Desenvolvimento dos Osciladores em Anel}
\label{sec:MetOscilador}

A topologia de oscilador em anel foi escolhida para os testes por sua disseminada utilização na caracterização de dispositivos MOSFET. Seu uso é amplo, pois medidas utilizando esse circuito se aproximam muito mais de aplicações reais do que medições paramétricas DC padrões.

Foram realizados testes preliminares com diferentes quantidades de inversores para encontrar uma quantidade apropriada, pois, com poucos osciladores não há tempo suficiente para os inversores chavearem e com muitos osciladores o limite de iterações em um bloco 'for' que as IDEs permitem era atingido.

Considerando isso, foi decidido que em cada um dos dispositivos seria sintetizado dois osciladores, um com 1001 inversores e outro com 4999. A escolha de utilizar dois osciladores em cada FPGA foi tomada por dois motivos: para se ter certeza que as IDEs não estavam simplificando os estágios inversores do circuito sintetizado e para verificar que o envelhecimento afeta igualmente diferentes partes do FPGA.

A grande quantidade de inversores é relevante, pois assim as pequenas variações aleatórias nas características de cada transistor que compõe os dispositivos tenderão a se diluir.

\subsection{Desenvolvimento do Código para os Osciladores}

Para desenvolver e sintetizar os osciladores em anel foi utilizada a linguagem de descrição Verilog. Para o Cyclone II foi utilizada a IDE Quartus II versão 12.1, já para o ZedBoard foi utilizada a IDE Vivado versão 2023.1.

O Quadro \ref{code:RingOsc} mostra o código desenvolvido em Verilog para o módulo que implementa o oscilador em anel com N inversores, sendo N igual a 5 caso nenhum valor seja definido durante a instanciação do módulo. O mesmo código foi utilizado para os dois FPGAs nas duas IDEs diferentes.

\begin{lstlisting}[label={code:RingOsc}, style=VerilogStyle, caption={Módulo do Oscilador em Anel. Fonte: O Autor}]
module RingOscillator 
	#(parameter N = 5)
	(
		input  en,
		output reg and_1    /*synthesis keep*/
	);
	reg [N - 1:0] notGate /*synthesis keep*/;
	integer i;
	generate
	  always @ (*) begin
		  and_1 <= en & notGate[N - 1];
	  	notGate[0] <= ~and_1;
	  	for (i = 1; i < N; i = i + 1)   begin: inverter_chain
			  notGate[i] <= ~notGate[i - 1];
		  end
	  end
	endgenerate
endmodule
\end{lstlisting}

O módulo possui uma entrada en, responsável por habilitar o circuito, uma saída and\_1 que é a saída da porta and do circuito, de onde o sinal do oscilador é obtido. O valor N é parametrizável, o que permite a reutilização do mesmo código para osciladores com diferentes números de inversores.

São então instanciadas N variáveis, que serão utilizados para criar os inversores. A variável and\_1 recebe o resultado da operação E lógica do sinal de \textit{enable} e a saída do último inversor. O primeiro inversor é definido como a variável and\_1 invertido.

Um bloco 'for' é utilizado para automatizar as atribuições dos inversores seguintes, sendo a cada um atribuído o valor inversor anterior negado.

Um ponto importante de destacar é a necessidade de utilizar diretivas de compilação para impedir que os inversores sejam simplificados na síntese. Essas diretivas são diferentes em cada uma das IDEs, na Quartus II é utilizado a diretiva /* synthesis keep */ e na Vivado é utilizado a diretiva /* synthesis syn\_keep=1 */.

% Na Figura \ref{fig:DE2Imp3Osc} pode ser visto como o Quartus II implementa em hardware o código do Quadro \ref{code:RingOsc} para um N igual a 3. A implementação é feita através de portas lógicas comuns, o que é diferente da implementação feita pelo Vivado, como visto na Figura \ref{fig:ZedImp3Osc}, que utiliza LUTs de uma variável para representar os inversores.

Nas Figuras \ref{fig:DE2Imp3Osc} e \ref{fig:ZedImp3Osc} pode ser visto como, respectivamente, o Quartus II e o Vivado implementam em hardware o código do quadro \ref{code:RingOsc} para um N igual a 3. É importante destacar que, por mais que para o Quartus II sejam utilizados portas lógicas para ilustrar o circuito, essas portas lógicas são construídas, na verdade, com LUTs que implementam o seu comportamento.

\begin{figure}[H]
    \centering
    \includegraphics[width=\linewidth]{figures/Metodologia/DE2_Implementation_3Inverter_Gates.png}
    \caption{Síntese de um oscilador com três inversores no Quartus II. Fonte: O Autor}
    \label{fig:DE2Imp3Osc}
\end{figure}

\begin{figure}[H]
    \centering
    \includegraphics[width=\linewidth]{figures/Metodologia/ZedBoard_Implementation_3Inverter.png}
    \caption{Síntese de um oscilador com três inversores gerado no Vivado. Fonte: O Autor}
    \label{fig:ZedImp3Osc}
\end{figure}

O Quadro \ref{code:TopLevel} mostra o módulo de alto nível em que é instanciado dois osciladores em anel, um com 1001 e outro de 4999 inversores. 

\begin{lstlisting}[label={code:TopLevel}, style=VerilogStyle, caption={Instanciação dos Módulos. Fonte: O Autor}]
module TopLevel
	(
		input en,
		output run,
		output out1001, out4999
	);
	
	assign run = en;

	RingOscillator #(.N(1001)) ring1001(en, out1001);
	RingOscillator #(.N(4999)) ring4999(en, out4999);
endmodule
\end{lstlisting}

O módulo possui uma entrada 'en', responsável por habilitar o circuito, uma saída 'run', usada para indicar que o circuito está em funcionamento e as saídas dos dois osciladores 'out1001', 'out4999'. Também são declaradas duas instâncias do módulo desenvolvido no Quadro \ref{code:RingOsc}.

As Figuras \ref{fig:DE2RtlSchem} e \ref{fig:ZedRtlSchem1} mostram, respectivamente, o circuito implementado pelo software Quartus II e Vivado para o módulo de alto nível que serão utilizado nos FPGAs.

\begin{figure}[H]
    \centering
    \includegraphics[scale=0.25]{figures/Metodologia/DE2_RTL_Schematic.png}
    \caption{Síntese do módulo de alto nível gerado no Quartus II. Fonte: O Autor}
    \label{fig:DE2RtlSchem}
\end{figure}

\begin{figure}[H]
    \centering
    \includegraphics[width=\linewidth]{figures/Metodologia/ZedBoard_RTL_Schematic.png}
    \caption{Síntese do módulo de alto nível gerado no Vivado. Fonte: O Autor}
    \label{fig:ZedRtlSchem1}
\end{figure}

Nas duas placas a entrada 'en' foi atribuída a um pino do FPGA ligado a uma chave, a saída 'run' foi atribuída a um pino do FPGA ligado a um LED e as saídas 'out1001' e out4999 foram atribuídas a pinos do FPGA ligados a conectores de entrada e saída de uso geral.

Após o desenvolvimento, o circuito sintetizado foi simulado utilizando a ferramenta apropriada para cada um dos componentes. Constatada a validade da solução ela foi transferida para os FPGAs reais e será medida, através de um osciloscópio, a frequência de oscilação das saídas do circuitos.
\section{Ensaios de Envelhecimento}
\label{sec:MetEnsaios}

Para induzir o fenômeno de BTI nos dispositivos foi utilizada uma câmara térmica para ensaios com componenentes eletrônicos. Inicialmente a temperatura de exposição foi de 100°C, temperatura que é inferior a faixa de ocorrência do BTI, mas que foi utilizada para verificar se os componentes não iriam ser danificados com a temperatura.

A Figura \ref{fig:CamTerm} mostra a câmera térmica utilizada para os ensaios. Ele foi fabricado pela SPX Thermal Product Solutions sendo capaz de alcançar temperaturas de -75ºC a 200ºC.

\begin{figure}[H]
    \centering
    \includegraphics[angle=270, scale=0.08]{figures/Metodologia/Ensaios_CamaraTermica.jpg}
    \caption{Câmara térmica utilizada para os ensaios. Fonte: O Autor}
    \label{fig:CamTerm}
\end{figure}

Constatando que não houve problema à 100ºC, a temperatura foi aumentada para 125°C. Como também não houve danos, a temperatura foi, então, elevada para 135°C.

Nessa temperatura foi observado que o conector de alimentação da placa DE2 apresentava sinais de derretimento, por isso, a temperatura dos ensaios foi definida em 135°C. Temperatura que está dentro da faixa de 125 e 175ºC que a bibliografia indica como sendo a faixa em que o BTI ocorre mais facilmente.

O tempo total de estresse térmico foi de 150h, tempo definido levando-se em conta o que foi estudado na seção \ref{sec:GerandoNbti}. A contagem do tempo era iniciada quando a câmara térmica era ligada e paralisada quando desligada. 

O tempo que os dispositivos foram expostas ao calor não foi contínua, devido a impossibilidade de ficar durante a noite no laboratório e, por motivos de segurança, de deixar a câmara térmica ligada sem supervisão. Portanto os dispositivos, de forma geral, foram expostos à câmara térmica durante o dia e retirados dela durante a noite, ficando ligados o tempo todo, de forma que não houvesse relaxamento.

As 150h de exposição foram divididas ao longo de 19 dias, não sendo necessariamente dias seguidos. A Tabela \ref{tab:TempoDia} mostra quanto tempo as placas foram expostas em cada dia em que foi realizado ensaio.

\begin{table}[htp]
\centering
\caption{Tempo de exposição por dia}
\begin{tabular}{|c|c|}
\hline
\multicolumn{1}{|c|}{\textbf{Dia}} & \textbf{Tempo de Exposição (h)} \\ \hline
18/10/2023 & 4,63 \\
19/10/2023 & 4,17 \\
20/10/2023 & 5,17 \\
23/10/2023 & 10,05 \\
24/10/2023 & 9,72 \\
25/10/2023 & 8,25 \\
30/10/2023 & 9,25 \\
01/11/2023 & 6,28 \\
03/11/2023 & 7,25 \\
08/11/2023 & 9,25 \\
10/11/2023 & 7,28 \\
16/11/2023 & 8,13 \\
17/11/2023 & 8,45 \\
22/10/2023 & 9,83 \\
23/11/2023 & 8,17 \\
29/11/2023 & 8,13 \\
30/11/2023 & 10,32 \\
11/12/2023 & 9,25 \\
12/12/2023 & 6,70 \\
13/12/2023 & 1,73 \\ \hline
\end{tabular}
\label{tab:TempoDia}
\end{table}

Outras duas placas, dos mesmo modelos das ensaiadas, foram deixadas fora da câmara térmica pelo mesmo tempo, sempre ligadas e com os mesmos osciladores em anel sintetizados. Comparar as medidas entre as placas que foram aquecidas com as que não foram é importante para verificar que se a degradação na frequência é proveniente do estresse térmico ou se é apenas resultado do funcionamento prolongado.

A câmara térmica permite realizar medidas nos dispositivos ensaiados enquanto eles estão dentro dela. Portanto foi possível medir a frequência dos osciladores em anel durante o processo de estresse térmico. As Figuras \ref{fig:CamTerm} e \ref{fig:Osciloscopio} mostram o setup experimental construído, com os FPGAs ensaiados dentro da câmera térmica com os cabos de medição saindo da câmera e ligadas ao osciloscópio.

\begin{figure}[H]
    \centering
    \includegraphics[scale=0.08]{figures/Metodologia/Ensaios_FpgasNoForno.jpg}
    \caption{FPGAs dentro da Câmara térmica. Fonte: O Autor}
    \label{fig:CamTerm}
\end{figure}

\begin{figure}[H]
    \centering
    \includegraphics[scale=0.08]{figures/Metodologia/Ensaios_Osciloscopio.jpg}
    \caption{Osciloscópio utilizado para as medidas. Fonte: O Autor}
    \label{fig:Osciloscopio}
\end{figure}

Enquanto a câmara aquecia da temperatura ambiente para a temperatura alvo, as medições foram mais frequentes, mantendo-se, no máximo um período de cinco minutos entre as medidas. Quando a temperatura alvo é atingida as medidas ficam mais esparsas, considerando que há menos variação na frequência medida.

É importante separar os efeitos instantâneos da temperatura na frequência dos osciladores do efeito a longo prazo do BTI. Por isso é necessário comparar as medidas feitas em uma mesma temperatura. O momento que a temperatura é a mais estável é quando a câmara já chegou aos 135°C.

Com as medidas ao longo do aquecimento da câmara é possível construir curvas de frequência por temperatura e compará-las para diferentes ciclos. Como em cada ciclo os dispositivos estão há um determinado tempo sofrendo estresse, será possível analisar o deslocamento dessa curva, se houver, com o passar do tempo.

Comparando essas medidas em relação ao tempo total estressado será possível constatar se houve ou não uma degradação na frequência de funcionamento dos dispositivos, e se houve uma diferença significativa na degradação entre as duas placas.

Após os ensaios de envelhecimento foi realizado um ensaio para medir a recuperação dos circuitos, para verificar quanto da degradação sofrida é reversível. Para isso as quatro placas foram desligadas e deixadas em temperatura ambiente. Medidas ao longo do tempo foram realizadas até o tempo de relaxamento totalizar o tempo total que as placas foram estressadas.


% \chapter{Resultados}
\label{sec:Resultados}

Este capítulo irá mostrar os resultados obtidos através dos dados colhidos durante os ensaios de envelhecimento detalhados na seção \ref{sec:MetEnsaios}. Serão apresentadas as curvas de como a frequência dos osciladores variou com o tempo de exposição ao calor.

\section{Medidas Iniciais}
\label{sec:ResMedidasIniciais}

As primeiras medidas foram realizadas em temperatura ambiente antes dos ensaios iniciarem, para se ter as frequências iniciais dos osciladores. A Tabela \ref{tab:FreqIniciais} mostra esses valores os dois osciladores de cada placa. Esses valores serão utilizados como valor unitário quando os valores das frequências medidas estiverem normalizado.

\begin{table}[htp]
\centering
\caption{Frequências iniciais dos osciladores.}
\begin{tabular}{|cc|cc|}
\hline
\multicolumn{2}{|c|}{\textbf{Altera DE2}} & \multicolumn{2}{c|}{\textbf{ZedBoard}} \\ \hline
\multicolumn{1}{|c|}{\textbf{1001}} & \textbf{4999} & \multicolumn{1}{c|}{\textbf{1001}} & \textbf{4999} \\ \hline
\multicolumn{1}{|c|}{1805kHz} & 364,9kHz & \multicolumn{1}{c|}{1470kHz} & 276,5kHz \\ \hline
\end{tabular}
\label{tab:FreqIniciais}
\end{table}

A frequência dos osciladores da placa ZedBoard foram menores que os da placa DE2, mesmo o nó tecnológico dela sendo menor. Isso pode ser devido ao fato das LUTs que constituem o circuito sintetizado na ZedBoard serem mais complexos que os presentes na DE2.
\section{Comparação das Curvas à temperatura ambiente}
\section{Comparação das Curvas a 135ºC}

\begin{figure}[H]
    \centering
    \includegraphics[scale=0.75]{figures/Resultados/T135DE2}
    \caption{Curva da DE2 a 135ºC. Fonte: O Autor}
    \label{fig:T135DE2}
\end{figure}

\begin{figure}[H]
    \centering
    \includegraphics[scale=0.75]{figures/Resultados/T135ZedBoard}
    \caption{Curva da ZedBoard a 135ºC. Fonte: O Autor}
    \label{fig:T135ZedBoard}
\end{figure}

\begin{figure}[H]
    \centering
    \includegraphics[scale=0.75]{figures/Resultados/T135Ambas}
    \caption{Comparação das duas placas a 135ºC. Fonte: O Autor}
    \label{fig:T135Ambas}
\end{figure}
\section{Comparação das Curvas Frequência X Temperatura}

\begin{figure}[H]
    \centering
    \includegraphics[scale=0.75]{figures/Resultados/FreqXTempDE21001}
    \caption{Curva Freqência por Temperatura do oscilador com 1001 inversores da placa DE2. Fonte: O Autor}
    \label{fig:FreqXTempDE21001}
\end{figure}

\begin{figure}[H]
    \centering
    \includegraphics[scale=0.75]{figures/Resultados/FreqXTempZedBoard1001}
    \caption{Curva Freqência por Temperatura do oscilador com 1001 inversores da placa ZedBoard. Fonte: O Autor}
    \label{fig:FreqXTempZedBoard1001}
\end{figure}
\section{Comparação das Curvas de relaxamento}

% Cite as palavras-chave que necessariamente aparecerão no capítulo sobre Experimentos e Resultados do seu Projeto de Diplomação.
% Ensaios de envelhecimento, medição de frequência, medidas de performance.

% Explique que experimentos foram (e/ou ainda serão) realizados para a obtenção de resultados que sirvam à validação da solução implementada.
% O principal experimento realizado será a exposição dos dispositivos à envelhecimento utilizando câmara térmica disponível no Laboratório de Caracterização Elétrica. Cada um dos dois FPGAs terão dois osciladores com 1001 e 4999 inversores. Eles foram expostos a temperatura de 135°C por um tempo total de 150 horas. Foi possível medir a frequência durante a exposição ao calor com um osciloscópio, obtendo-se valores de frequência para diversas temperaturas.

% Liste os materiais, métodos e ferramentas efetivamente utilizados para a execução dos ensaios e experimentos. Relacione-os com os experimentos realizados.
% Osciloscópio - Usado para medir a frequência dos osciladores.
% Câmara Térmica - Usado para estressar os FPGAs.
% FPGAs - Objeto do estresse.

% Descreva brevemente os resultados mais importantes obtidos e sua aplicação no contexto do problema.
% Os resultados mais importantes são as mudanças nos valores de frequência ao longo do tempo de envelhecimento. Foi verificado que a frequência do FPGA da placa ZedBoard degradou consideravelmente mais que a frequência do FPGA da placa DE2. Também foi verificado que o comportamento da degradação desacelera com o tempo, o que é esperado pelo que existe na literatura.

% Quais das suas hipóteses iniciais foram completa ou parcialmente comprovadas com os resultados obtidos? Quais hipóteses não foram comprovadas?
% Foi observado uma maior degradação na frequência do FPGA da placa ZedBoard. Isso corrobora a hipótese de que as duas placas não envelhecem com a mesma intensidade, o que pode ser devido ao fato que o nó tecnológico da ZedBoard ser menor que o da DE2, o que está de acordo com a literatura, que diz que o fenômeno do NBTI é mais crítico quanto menor for o transistor.

% Compare os resultados obtidos e posicione a sua solução em relação aos trabalhos correlatos apresentados no capítulo da fundamentação teórico-prática.
% Houve uma degradação de entorno de 1\% para a DE2 e 2,5\% para a ZedBoard. Comparando com outros trabalhos pode-se dizer que esses valores estão dentro do esperado.
% O trabalho de (Lorenz, 2013) mostra uma degradação de 5\% com 144 horas de exposição à 125°C.
% Já o trabalho de (Sato et al., 2014), que estudou métodos para diminuir o efeito de NBTI em osciladores em anel, resultou uma degradação de 0,25\%, com 42 horas de exposição, porém à apenas 85°C.

% Formule um resumo do conteúdo pretendido para o capítulo de experimentos e resultados, incluindo elementos utilizados para responder às perguntas anteriores.
% O capítulo de experimentos e resultados conterá as curvas de frequência por tempo de envelhecimento dos FPGAs para uma temperatura constante (Temperatura ambiente e 135°C). Será discutido as diferenças entre as duas placas e sobre o formato da curva.
% Também contará com comparação entre as curvas de frequência por temperatura de uma placa para diferentes tempos de envelhecimento e feito uma comparação entre as duas placas, sendo que a DE2 não apresentou grande alteração, já a Zedboard sim.
% Por fim, haverá uma discussão se esses resultados estão de acordo com o que foi apresentado no capítulo de Referências Bibliográficas.











%\bibliographystyle{bib/ref}
\bibliography{bib/ref.bib}
\begin{listofabbrv}{M2(CN2)M}
        \item[FPGA] Field-Programmable Gate Array
        \item[MOSFET] Metal Oxide Semiconductor Field Effect Transistor
        \item[CMOS] Complementary Metal Oxide Semiconductor
        \item[PMOS] Positive Metal Oxide Semiconductor
        \item[NMOS] Negative Metal Oxide Semiconductor
        \item[BTI] Bias Temperature Instability
        \item[PBTI] Positive Bias Temperature Instability
        \item[NBTI] Negative Bias Temperature Instability
        \item[RD] Reaction Diffusion
        \item[IDE] Integrated Development Environment
        \item[TDDB] Time-Dependent Dielectric Breakdown
        \item[EM] Electromigration
        \item[HCI] Hot Carrier Injection
\end{listofabbrv}
%\input{chapters/07_appendix.tex

\end{document}
